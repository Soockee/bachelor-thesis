% ----------------------------------------------------------------------------
% Copyright (c) 2016 by Burkhardt Renz. All rights reserved.
% Die Vorlage für eine Abschlussarbeit in der Informatik am Fachbereich
% MNI der THM ist lizenziert unter einer Creative Commons
% Namensnennung-Nicht kommerziell 4.0 International Lizenz.
%
% Id:$
% ----------------------------------------------------------------------------

\chapter{Einleitung}


\section{Motivation}
	Die heutigen Bedürfnisse der Anwender ein stets erreichbaren, fehlerfreien und ??schnellen?? Service zur Verfügung zu haben, stellt hohe Erwartung an Unternehmen. Um den Ansprüchen der Nutzer gerecht zu werden, müssen Systeme gewährleisten, dass ein gewisser Grad von Beobachtbarkeit des Systems erreicht wird. Die Beobachtbarkeit sorgt für die nötige reaktionsfähigkeit der Entwickler und Operatoren, um möglicherweise auftretende Komplikationen, die die  Benutzerbedürfnisse beeinträchtigen, schnell, präzise und langfristig beheben zu können.
	
	Die Komplexität des Gesamtsystems, welches aus vielen kleinen Komponenten bestehen kann, ist eine große Herausforderung für Entwickler und Operatoren. Die enorme Skalierbarkeit einzelner Komponenten und die aussgezeichnete Resourcennutzung der Hardware löst zwar viele Probleme der Vergangenheit, wie zum Beispiel Überbelastung einzelner Knoten, Ausfall von Komponenten, Latenzprobleme. Allerdings schafft diese Umstellung neue Schwierigkeiten, die es zu bewältigen gilt.
	
	Glossarbeispiel \gls{latex}
	Glossarbeispiel \gls{fpsLabel}
\section{Problemstellung}
\section{Foschungsstand}
\section{Thesisübersicht}
% ----------------------------------------------------------------------------
