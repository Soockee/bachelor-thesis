% ----------------------------------------------------------------------------
% Copyright (c) 2016 by Burkhardt Renz. All rights reserved.
% Die Vorlage für eine Abschlussarbeit in der Informatik am Fachbereich
% MNI der THM ist lizenziert unter einer Creative Commons
% Namensnennung-Nicht kommerziell 4.0 International Lizenz.
%
% Id:$
% ----------------------------------------------------------------------------

\chapter{Einleitung}
\label{chapter:Einleitung}


\section{Motivation}
\label{section:Motivation}
	Die heutigen Bedürfnisse der Anwender ein stets erreichbaren, fehlerfreien und performanten Service zur Verfügung zu haben, stellt hohe Erwartung an Unternehmen und deren Entwickler und Operatoren. Um den Ansprüchen der Nutzer gerecht zu werden, müssen Systeme gewährleisten, dass ein gewisser Grad von Beobachtbarkeit des Systems erreicht wird. Die Beobachtbarkeit sorgt für die nötige Reaktionsfähigkeit der Entwickler und Operatoren, um möglicherweise auftretende Komplikationen, die die Benutzerbedürfnisse beeinträchtigen, schnell, präzise und langfristig beheben zu können.
	
	Die Komplexität des Gesamtsystems, welches aus vielen kleinen Komponenten bestehen kann, ist eine große Herausforderung für Entwickler und Operatoren. Die enorme Skalierbarkeit einzelner Komponenten und die ausgezeichnete Ressourcennutzung der Hardware löst zwar viele Probleme der Vergangenheit, wie zum Beispiel Überbelastung einzelner Knoten, Ausfall von Komponenten und Latenzprobleme. Allerdings schafft diese Umstellung neue Schwierigkeiten, die es zu bewältigen gilt. 
	
	Die Instrumentalisierungsbibliothek \emph{Traktor} soll Events innerhalb eines Systems generieren und ordnen, sodass während der Entwicklungsphase eines Systems Fehlerquellen lokalisiert werden können. Bei der Entwicklung der Instrumentalisierungsbibliothek sollen Standards ermittelt, analysiert und umgesetzt werden. Es sollen Erfahrungswerte in der Domain der Beobachtbarkeit von Systemen, durch die Entwicklung einer Instrumentalisierungsbibliothek, gewonnen werden.
	
\section{Problemstellung}
\label{section:Problemstellung}
	Im Rahmen dieser Bachelorarbeit sollen Events in einem verteilten Rendering-System generiert und untersucht werden. Die Kommunikation innerhalb eines verteilte Systems über Prozessgrenzen spielt eine zentrale Rolle.
	Die Intrasystem - Netzwerkkommunikation bedarf Konzepte zur Nachvollziehbarkeit von Events und deren Beziehungen zueinander. 
	
	Das verteilte Rendering-System generiert asynchron \emph{Frames} und sendet diese in Intervallen über eine Websocketverbindung an einen Client. Frames werden verworfen, sobald neuere Frames generiert wurden, d.h. innerhalb eines Intervals können mehrere Frames entstehen, aber nur eines ist relevant. Die Einhaltung von zeitlichen Grenzwerten, wie z.B. die Framegenerierungszeit oder der End-zu-End Latenz, gilt zu überprüfen. Der Klient kann asynchron Einfluss auf die zu generierenden Frames durch Übermittlung von Daten nehmen. Die bei diesem Datenaustausch entstehenden Events wie z.B. das Starten einer Framegenerierung, dem Beenden einer Framegenerierung, dem Senden eines fertiggestellten Frames und dem Empfangen eines Frames sollen erstellt werden. Die zeitliche Einordnung der Events hat dabei eine zentrale Rolle einzunehmen. Die Eventgenerierung muss sich mit den zeitlichen Rahmenbedingungen vereinbaren lassen. Das Konstrukt von Events die zueinander in Verbindung stehen, soll untersucht werden. Dazu stellt sich folgende Frage: \emph{Ist es möglich Events in einem verteilten System zu generieren und miteinander in Verbindung zu setzen, die es erlauben, einen Stream von Frames als eine Anordnung von Events, die kausal miteinander verbunden sind, darzustellen?}
	
\section{Foschungsstand}
\label{section:Forschungsstand}

Es gibt diverse Konzepte und Werkzeuge zur Erhebung von Tracingdaten. Darunter zählen \emph{state-of-the-art} Tracinginfrastrukturen, Instrumentalisierungsbibliotheken und Open-Soruce \gls{apiLabel} Spezifikationen wie z.B.:
\begin{itemize}
	\item Zipkin
	\item Jaeger
	\item Opentracing
\end{itemize}

Publikationen, auf die sich diese Werkzeuge beziehen, sind unter anderem: 

\begin{itemize}
	\item Dapper \footpartcite{Shanbhag2010}
	\item Pivot Tracing \footpartcite{10.1145/2815400.2815415}
	\item X-Trace \footpartcite{Stocia2007}
	\item X-Ray \footpartcite{10.5555/2387880.2387910}
	\item Canopy \footpartcite{Kaldor2017}
\end{itemize}
Das Paper \emph{End-to-End Tracing Models: Analysis and Unification}\footpartcite{Leavitt2014} beschreibt das spanbasierte Modell als eine Sammlung von \emph{spans}, welche jeweils eine Block von Rechenarbeit darstellen. die Pulibkation \emph{Dapper} ist von Google veröffentlicht und präsentiert ihr entwickletes Konzept zur Überwachung von Events in ihrer extremen Umgebung. Die Brown University präsentiert in ihrer Veröffentlichung \emph{Universal Context Propagation for Distributed System Instrumentation}\footpartcite{10.1145/3190508.3190526} eine Schichten-Architektur zur Übermittlung von Tracingdaten in einem spezifizierten \emph{Baggage Context}. Der Baggage Context wird als Metadatei mit gereicht und stellt somit eine Form des Piggybacking dar. Canopy ist die von Facebook entwickelte Tracinginfrastruktur, welches einen Fokus auf Echt-Zeit Analyse der Performancedaten legt.

\section{Thesisübersicht}
\label{section:Thesisübersicht}

Diese Bachelorarbeit beinhaltet folgende Kapitel:

\begin{itemize}
	\item \textbf{Kapitel 2} liefert einen kurzen Überblick über relevantes Grundwissen. Dabei wird auf verteilte Systeme und deren Beobachtbarkeit eingegangen. Zudem wird ein Einstieg in Distributed Tracing geboten.
	\item \textbf{Kapitel 3} beschreibt die Problemstellung und stellt drei Forschungsfragen auf. Anschließend wird eine Anforderungsanalyse durchgeführt.
	\item \textbf{Kapitel 4} stellt das Design von Traktor vor. Es werden Designziele definiert. Konzepte und Spezifikationen werden vorgestellt. 
	\item \textbf{Kapitel 5} beinhaltet die Implementierung. Sie beschreibt die Implementierungsdetails der Komponenten der Tracinginfrastruktur. Auch die Anwendungsfälle der Bibliothek werden vorgestellt.
	\item \textbf{Kapitel 6} beinhaltet die Evaluation. Es wird die Anforderungs- und Designerfüllung diskutiert. Die Bereitstellung der Infrastruktur wird diskutiert. Ein Ergebnis -und Darstellungsvergleich mit einer bewährten Tracingwerkzeug wird durchgeführt.
	\item \textbf{Kapitel 7} zieht das Fazit der Arbeit. Zudem werden die Limitierungen beschrieben. Zuletzt wird ein  Ausblick gegeben.
\end{itemize}
% ----------------------------------------------------------------------------
