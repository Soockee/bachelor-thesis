% ----------------------------------------------------------------------------
% Copyright (c) 2016 by Burkhardt Renz. All rights reserved.
% Die Vorlage für eine Abschlussarbeit in der Informatik am Fachbereich
% MNI der THM ist lizenziert unter einer Creative Commons
% Namensnennung-Nicht kommerziell 4.0 International Lizenz.
%
% Id:$
% ----------------------------------------------------------------------------

\chapter{Einleitung}
Oftmals stellt man sich die Frage, mit welchen Werkzeugen eine Aufgabe am effizientesten gelöst werden kann. Insbesondere Informatiker sehen sich mit dem Problem konfrontiert, einer geradezu endlosen, sich stetig wandelnden Auswahl an Programmiersprachen, Konzepten, Algorithmen und dergleichen, gegenüberzustehen.\\
Diese Arbeit soll sich mit den bewährten Programmiersprachen C und R beschäftigen. Unter Verwendung dieser Sprachen, werden verschiedene Ressourcen untersucht, die gebraucht werden, um ein komplexes mathematisches Problem zu lösen.

\section{Zielsetzung}
Das Ziel dieser Arbeit soll sein, ein Rechenzeitvergleich der Programmiersprachen C und R durchzuführen. C ist bekannt dafür, eine Hardwarenahe und schnelle Programmiersprache zu sein. Als Hypothese wird aufgestellt, dass C bessere Rechenzeiten aufweist, als R. Als Vergleichsgrundlage soll die Berechnung des Inversen einer großen Matrix dienen.


% ----------------------------------------------------------------------------
