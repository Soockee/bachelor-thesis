% ----------------------------------------------------------------------------
% Copyright (c) 2016 by Burkhardt Renz. All rights reserved.
% Die Vorlage für eine Abschlussarbeit in der Informatik am Fachbereich
% MNI der THM ist lizenziert unter einer Creative Commons
% Namensnennung-Nicht kommerziell 4.0 International Lizenz.
%
% Id:$
% ----------------------------------------------------------------------------

\chapter{Themenüberblick}

\section{Verteilte Systeme}

Verteilte Systeme dienen als Anwendungsfeld für die Instrumentalisierungsbibliothek. Auch das in dieser Arbeit vorgestellte Testbett ist ein solches verteiltes System. Dazu ist es wichtig,  verschiedene Eigenschaften und Konzepte von verteilten Systemen zu definieren. Diese Eigenschaften und Konzepte nehmen einen entscheidenden Faktor bei der Ermittlung, Analyse und Umsetzung von Anforderungen der Instrumentalisierungsbibliothek ein. Demstsprechenend wird zunächt der Begriff des verteilten Systems definiert:

\begin{quote}
	Ein verteiltes System ist eine Kollektion unabhängiger Computer, die den Benutzern als ein Einzelcomputer erscheinen\footpartcite{10.5555/1202502} 
\end{quote}

\subsection{Eigenschaften eines verteilten Systems}
	Die Definition von van Steen und Tanenbaum geht mit zwei charakteristischen Eigenschaften von verteilten Systemen einher. Die erste Eigenschaft äußert sich darin, dass alle Komponenten eines Systems unabhängig voneinander agieren können. Komponenten werden in diesem Zusammenhang auch Knoten genannt. Knoten sind Hardwarekomponenten, also pysikalische Recheneinheiten. Auch Prozesse innerhalb einer Hardwareeinheit können Knoten sein. Dabei ist es möglich, dass mehrere Knoten auf einer Hardwareheinheit sind.\footpartcite[p. 2]{VanSteen2017}.

	Diese Knoten kommunizieren mittels Nachrichten miteinander. 
\subsection{Beobachtbarkeit von verteilten Systemen}
	Die Beobachtbarkeit von verteilten Systemen kann in erster Linie durch Überwachung diverser Komponenten des Systems ermöglicht werden. Dabei lassen sich diese Komponenten in drei Kategorien unterteilen (\lowroman{1}) \textbf{Hardware}, (\lowroman{2}) \textbf{Netwerke} und (\lowroman{3}) \textbf{Anwendungen}

	
	\textbf{Hardware}\space\space\space some text
	
	\begin{itemize}
		\item CPU
		\item GPU
		\item Speicher
	\end{itemize}
	
	\textbf{Netzwerk}\space\space\space some text
	
	\begin{itemize}
		\item Anzahl Packete
		\item Latenz zwischen Komponenten bzw. Client
		\item Übertragungsanomalien
	\end{itemize}
	
	\textbf{Anwendung}\space\space\space some text
	
	\begin{itemize}
		\item Logs
		\item Metriken
		\item Graphen etc. 
	\end{itemize}

Quelle beziehen:
\href{https://de.wikipedia.org/wiki/Verteiltes_System}{Verteilte Systeme Zitate}
\subsection{Synchronisation}
\begin{itemize}
	\item 
	\item 
	\item
\end{itemize}
\subsection{Ordnung von events}
\begin{itemize}
	\item Bezug auf Lamports Eventordnung\footpartcite{lamport78}
\end{itemize}
\section{Bibliotheksentwicklung}
\begin{itemize}
	\item 
	\item 
	\item
\end{itemize}

\section{Distributed Tracing}
Verteilte Systeme wurden bereits definiert. Allerdings lohnt es sich eine alternative Definition zu betrachten.

\begin{quote}
	Ein verteiltes System ist ein System, mit dem ich nicht arbeiten kann, weil irgendein Rechner abgestürzt ist, von dem ich nicht einmal weiß, dass es ihn überhaupt gibt.\footpartcite{lamport87}
\end{quote}


Diese Definition lässt sich so auffassen, dass Lamport im Jahr 1987 die Problematik der Fehlersuche während der Laufzeit, des Debuggings während der Entwicklungsphase und des organisatorischen Aufwands im Allgemeinen, also damit der grundsätzlich hohen Unübersichtlichkeit und Komplexität von verteilten Systemen, beschreibt. Diese Probleme und Herausforderungen, mit denen man in der heutigen Zeit zusammenstößt, eventuell sogar noch intensiver, als zur damaligen Zeit, können durch distributed tracing angegangen werden. 


\begin{itemize}
	\item Beschreibe, was distributed tracing werkzeuge bezwecken
	\item Beschreibe, was blackbox bezweckt und warum es nicht das gleiche ist
	\item beschreibe, was metriken sind: bezug zu überwachung
	\item beschreibe was logs sind: bezug zu überwachung
	\item beschreibe was traces sind
	\item beschreibe ziele von distributed tracing
	\item beschreibe nicht-ziele von distributed tracing
	\item 
\end{itemize}
 
 
% ----------------------------------------------------------------------------
