% ----------------------------------------------------------------------------
% Copyright (c) 2016 by Burkhardt Renz. All rights reserved.
% Die Vorlage für eine Abschlussarbeit in der Informatik am Fachbereich
% MNI der THM ist lizenziert unter einer Creative Commons
% Namensnennung-Nicht kommerziell 4.0 International Lizenz.
%
% Id:$
% ----------------------------------------------------------------------------

\chapter{Problembeschreibung}
\label{chapter:Problembeschreibung}
\section{Eventgenerierung}
\label{section:Eventgenerierung}


\begin{figure}[!ht]
	\centering
	\includegraphics[scale=0.5]{img/synchronisation/distributed_system_application_minimal.png}
	\caption[Minimale Struktur eines verteilten Systems]{Minimale Struktur eines verteilten Systems, bestehend aus zwei Komponenten}
	\label{fig:distributed_system_application_minimal}
\end{figure}

Wir definieren ein minimales verteiltes System, bestehend aus zwei Komponenten. \cref{fig:distributed_system_application_minimal} bildet ein solches System ab. Die Knoten beinhalten zwei für das Generieren und Ordnen von Events interessante Aspekte. Dies ist zum einen die in \cref{fig:distributed_system_application_inside} dargestellte verteilte Anwendung mit ihrem instrumentalisiertem Code und zum Anderen das in \cref{fig:distributed_system_network} dargestellte Netzwerk, über welches Nachrichten ausgetauscht werden. 

\begin{figure}[!ht]
	\centering
	\begin{subfigure}[t]{.49\linewidth}
		\centering\includegraphics[width=0.9\linewidth]{img/synchronisation/distributed_system_application_inside.png}
		\caption[Abbildung]{zeigt Knoten mit instrumentalisiertem Anwendungscode}
		\label{fig:distributed_system_application_inside}
	\end{subfigure}
	\begin{subfigure}[t]{.49\linewidth}
		\centering\includegraphics[width=\linewidth]{img/synchronisation/distributed_system_network.png}
		\caption[Abbildung]{Netzwerkkommunikation über TCP/IP}
		\label{fig:distributed_system_network}
	\end{subfigure}
	\caption[Anwendungsinstrumentalisierung und Netzwerkkommunikation über TCP/IP in verteilten Systemen]{}
\end{figure} 

\subsection{Eventkorrelation}
\label{subsection:Eventkorrelation}
\subsection{Synchronistion von Eventgeneratoren}
\label{subsection:Synchronistion von Eventgeneratoren}
\section{Eventübermittlung}
\label{section:Eventübermittlung}

% ----------------------------------------------------------------------------
