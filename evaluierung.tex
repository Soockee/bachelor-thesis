% ----------------------------------------------------------------------------
% Copyright (c) 2016 by Burkhardt Renz. All rights reserved.
% Die Vorlage für eine Abschlussarbeit in der Informatik am Fachbereich
% MNI der THM ist lizenziert unter einer Creative Commons
% Namensnennung-Nicht kommerziell 4.0 International Lizenz.
%
% Id:$
% ----------------------------------------------------------------------------

\chapter{Evaluierung}
\label{chapter:Evaluierung}
In diesem Kapitel werden die Prozesse beschrieben, die durchgeführt wurden, um die Implementierung und die erhaltenen Ergebnisse zu evaluieren. In \cref{section:Anforderungserfüllung} wird die Traktorimplementierung auf ihre Einhaltung der Anforderungen  untersucht. In \cref{section:Umsetzung der Designziele} werden die Designziele herangezogen und auf ihre Gegebenheit in der Tracingbibliothek überprüft. Das Open-Source Projekt \textbf{Jaeger} wird als Vergleichswerkzeug in den Folgenden Abschnitten herangezogen. Jaeger ist eine auf der OpenTracing API basierenden \emph{state-of-the-art} Distributed Tracing Implementierung. Sie setzt die aktuellsten Erkenntnisse der Distributed Tracing Gemeinschaft um. Dabei wird in \cref{section:Bereitstellung der Testumgebung} die Bereitstellung der Testumgebung, in Hinsicht auf beide Werkzeuge, diskutiert. Die Bereitstellungsunterschiede beider Werkzeuge werden aufgezeigt. In \cref{section:Ergebnissvergleich} werden die Ergebnisse der Spangenerierung verglichen. Es wird auf die Ausdruckskraft des Traktor-Datenmodells im Vergleich zu Jaeger eingegangen. Zuletzt werden die präsentierten Visualisierungansätze diskutiert. Es wird ein Vergleich zu den Visualisierungsmöglichkeiten der Jaeger UI durchgeführt.

\section{Anforderungserfüllung}
\label{section:Anforderungserfüllung}

Es ist eine Analyse durchzuführen, bei der die erhobenen Daten interpretiert werden. Die daraus gewonnenen Informationen sollen die End-zu-End Latenz einer Anfrage durch ein verteiltes System und die Generierungszeit eines Frames, welches durch die Unity Anwendung generiert wurde, darstellen.
\begin{itemize}
	\item Funktionalitäten
		\begin{itemize}
			\item Eventgenerierung
			\item Eventrelation
			\item Synchronisation von Eventgeneratoren
			\item Eventübermittlung
			\item Ordnung von Events
		\end{itemize}
	\item End-zu-End Latenz
	\item Generierungszeit eines Frames
	\item Nachrichtenmodifikation
\end{itemize}

\section{Umsetzung der Designziele}
\label{section:Umsetzung der Designziele}
\section{Bereitstellung der Testumgebung}
\label{section:Bereitstellung der Testumgebung}
\begin{itemize}
	\item Durch Docker-Compose wird eine verteiltes System auf einem Lokalen Rechner aufgebaut.
	\item gibt es architektonische unterschiede? Services etc. 
	\item Konfigurationsunterschiede
	\item Ist traktor einfacher zu deployen? Wahrschienlich nicht da jaeger ein all in one image hat. Könnte aber auch umgesetzt werden
\end{itemize}
\section{Ergebnissvergleich}
\label{section:Ergebnissvergleich}


\begin{itemize}
	\item Bezug zur Problemstellung schaffen
	\item inwiefern hält sich jaeger an die Happens before relationship?
	\item wie setzt jaeger diese um?
	\item setzt es sie um?
	\item setzt traktor sie um? Wenn ja wie?
	\item Wie regelt Jaeger die Zeitresolution?
	\item  
\end{itemize}
\section{Visualisierungvergleich von Traktor und Jaeger}
\label{section:Visualisierungvergleich von Traktor und Jaeger}
\begin{itemize}
	\item jaeger bietet verschiedene visualisierungsmöglichkeiten
	\begin{itemize}
		\item z.B. DAG
		\item Tracediagramm
		\item Service-Orientierter Ansatz?
		\item Tracecompare
	\end{itemize}
	\item Welche Vorteile bringen meine Visualisierungsansatze?
	
\end{itemize}
% ----------------------------------------------------------------------------
