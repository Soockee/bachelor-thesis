% ----------------------------------------------------------------------------
% Vorlage Abschlussarbeit Informatik THM (minimal)
%
% Copyright (c) 2016 by Burkhardt Renz. All rights reserved.
% Die Vorlage für eine Abschlussarbeit in der Informatik am Fachbereich
% MNI der THM ist lizenziert unter einer Creative Commons
% Namensnennung-Nicht kommerziell 4.0 International Lizenz.
%
% $Id: vorlage.tex 3835 2016-09-26 09:17:05Z br $
% ----------------------------------------------------------------------------

\documentclass[%
	BCOR=8.25mm,         % Bindekorrektur
	DIV=12,              % Satzspiegel
	parskip=half,				 % Abstand zwischen Absätzen
	bibliography=totoc,	 % Literaturverzeichnis im Inhaltsverzeichnis
	headsepline=on,      % Trennlinie Kolumnentitel
	openany,
	ngerman
	]{scrbook}

%% Präambel
\usepackage[english, ngerman]{babel} % deutsche typogr. Regeln + Trenntabelle
\usepackage[T1]{fontenc}             % interner TeX-Font-Codierung
\usepackage{lmodern}                 % Font Latin Modern
\usepackage[utf8]{inputenc}          % Font-Codierung der Eingabedatei
\usepackage[babel]{csquotes}         % Anführungszeichen
\usepackage[figurename=Abbildung]{caption}
\usepackage{graphicx}                % Graphiken
\usepackage{booktabs}                % Tabellen schöner
\usepackage{listingsutf8}            % Listings mit Einstellungen
\usepackage{chngcntr}
\counterwithout{footnote}{chapter}   % Document Wide cite index
\lstset{basicstyle=\small\ttfamily,
	tabsize=2,
	basewidth={0.5em,0.45em},
	extendedchars=true}
\usepackage{amsmath}	            % Mathematik
\usepackage{enumitem}				% Listen Aufzählung
\usepackage[section]{placeins}
\makeatletter						% Expand placeins function to subsection
\AtBeginDocument{%
	\expandafter\renewcommand\expandafter\subsection\expandafter{%
		\expandafter\@fb@secFB\subsection
	}%
}
\makeatother
\usepackage{geometry}
%\geometry{a4paper
%	,left=40mm, right=15mm, top=20mm, bottom=15mm   %Seitenr"ander
	%,body={11cm,17cm}
	%,textheight=17cm,textwidth=11cm
%	,headsep=1cm      %Abstand Seitenzahl - Text
%	,showframe=true
%}
\usepackage[english]{babel}
\usepackage[backend=biber,maxnames=999,style=alphabetic]{biblatex}
\addbibresource{bib/library.bib} 
\usepackage{scrhack}								 % unterdrückt Fehlermeldung von listings
%% Nummerierungstiefen
\setcounter{tocdepth}{3}             % 3 Stufen im Inhaltsverzeichnis
\setcounter{secnumdepth}{3} 		     % 3 Stufen in Abschnittnummerierung


 
%% Literaturverzeichniss Format
% Adapted from \footcite in numeric.cbx and generic citation
% commands \citeauthor, \citetitle, \citeyear in biblatex.def
\DeclareCiteCommand{\footpartcite}[\mkbibfootnote]
{\usebibmacro{prenote}}
{\usebibmacro{citeindex}%
	\mkbibbrackets{\usebibmacro{cite}}%
	\setunit{\addnbspace}
	\printfield[citetitle]{title}%
	\newunit
	\printfield{year}
}
{\addsemicolon\space}
{\usebibmacro{postnote}}

\DeclareMultiCiteCommand{\footpartcites}[\mkbibfootnote]{\footpartcite}{\addsemicolon\space}

%% Römische Nummer im Textverweden
\newcommand{\uproman}[1]{\uppercase\expandafter{\romannumeral#1}}
\newcommand{\lowroman}[1]{\romannumeral#1\relax}


%% Chapterschriftformatierung
\usepackage{titlesec, blindtext, color}
\definecolor{gray75}{gray}{0.75}
\newcommand{\hsp}{\hspace{20pt}}
\titleformat{\chapter}[hang]{\Huge\bfseries}{\thechapter\hsp\textcolor{gray75}{|}\hsp}{0pt}{\Huge\bfseries}

\usepackage[pdftex]{hyperref}       
\hypersetup{
	bookmarksopen=true,
	bookmarksopenlevel=3,
	colorlinks,
	citecolor=blue,
	linkcolor=blue,
}
\usepackage{cleveref}               % Smart Cross-Doc refs
\urlstyle{same}  % don't use monospace font for urls
%% Glossardefinition
\usepackage{subcaption} % Images horizontially
\usepackage[toc,acronym]{glossaries} % Glossarpaket
\makeglossaries
\renewcommand*{\acronymname}{Abkürzungsverzeichnis}
\newglossaryentry{latex}
{
	name={latex},
	description={Is a mark up language specially suited for scientific documents}
}

\newacronym{fpsLabel}{FPS}{Frames per Second}
\newacronym{MMULabel}{MMU}{Memory Mangagment Unit}
\newacronym{cpuLabel}{CPU}{Central Processor Unit}
\newacronym{gpuLabel}{GPU}{Graphical Processor Unit}
\newacronym{ramLabel}{RAM}{Random-Access Memory}
\newacronym{hddLabel}{HDD}{Hard Disk Drive}
\newacronym{tcpLabel}{TCP}{Transmission Control Protocol}
\newacronym{ipLabel}{IP}{Internet Protocol}
\newacronym{udpLabel}{UDP}{User Datagramm Protocol}
\newacronym{httpLabel}{HTTP}{Hypertext Transfer Protocol}
\newacronym{rfcLabel}{RFC}{Request for Comments}
\newacronym{iotLabel}{IoT}{Internet of Things}
\newacronym{apiLabel}{API}{Application Programming Interface}
\newacronym{cncfLabel}{CNCF}{Cloud Native Computing Foundation}
\newacronym{guidLabel}{GUID}{Globally Unique Identifier}
\newacronym{lfLabel}{LF}{Linux Foundation}
\newglossaryentry{cpuGlossar}{
	name={Zentrale Verarbeitungseinheit},
	description={Die zentrale Verarbeitungseinheit, im englischen auch \emph{Central Processor Unit} (CPU) genannt, ist die Komponente eines Systems, welches für die Abarbeitung des Maschinenprogramms zuständig ist. Im Allgemeinen Sprachgebraucht umfasst der Begriff der CPU mehrere Teilkomponenten, wie zum Beispiel dem Steuerwerk, der Register, des Speichermanagers (\gls{MMULabel})}}
\newglossaryentry{ciGlossar}{
	name={Continuous Integration},
	description={Automatisierter und fortlaufend getätigter Integrationsprozess von Änderungen einer Anwendungen.}}
\newglossaryentry{renderingGlossar}{
	name={Rendering},
	description={
	Rendering ist das erzeugen von zwei-dimensionalen Bildern.
	Rohdaten, wie z.B. virtuelle Kameras, drei-dimensionale Objekte, Lichtquellen, etc. werden dazu genutzt, um diese Bilder zu generieren}
}
\newglossaryentry{tddGlossar}{
	name={Test-driven development},
	description={
		Test-driven development ist ein Software Entwicklungskonzept. Kurze Iterationsphase stehen durch Testimplementierung im Vordergrund. Eine Testbasis senkt die technischen Schulden und erleichtert eine automatisierte Bereitstellung. 
	}
}
\newglossaryentry{clrGlossar}{
	name={Common Language Runtime},
	description={
		Die Common language Runtime ist eine virtuelle Laufzeitumgebung von .NET Anwendungen. 
	}
}


\pagenumbering{roman}
% ----------------------------------------------------------------------------
\begin{document}
\frontmatter

%% Titelseite
\begin{titlepage}
	\begin{center}
	\includegraphics[width=0.9\textwidth]{img/mni-logo}\\[5cm]
	\textbf{\huge\sffamily Generierung und Ordnung von Events in verteilten Systemen mit asynchroner Kommmunikation}\\[2cm]
	\textsc{\Large Bachlorthesis}\\Studiengang Informatik\\[2cm]
	vorgelegt von\\
	\textbf{Simon Stockhause}\\ [1.5cm] 
	Mai 2020
	\end{center}
	\vfill
	\center
	\begin{tabular}{ll}
		Referent der Arbeit: & Prof. Dr. Harald Ritz\\ 
		Korreferent der Arbeit: & M.Sc. Pascal Bormann\\ 
	\end{tabular}
\end{titlepage}
\cleardoubleemptypage

%% Erklärung
\pagestyle{empty}
{
	\renewcommand{\thispagestyle}[1]{}
 Hier Steht Erklärung
}
\clearpage
\pagestyle{plain}

%% Zusammenfassung
\pagestyle{empty}
\begin{quote}
	\vspace*{4cm}

	\begin{center}
		\textbf{\Large\sffamily Zusammenfassung}
	\end{center}
\end{quote}
\cleardoubleemptypage

%% Verzeichnissse
%% Kommando sorgt für das Entfernen der Seitennummerierung in römischen Zahlen 
%% bei Inhaltsverzeichniss und Abbildnugsverzeichniss

\pagestyle{empty}
{
	\renewcommand{\thispagestyle}[1]{}
	\tableofcontents
}
\clearpage
\pagestyle{plain}
\pagestyle{empty}
{
	\renewcommand{\thispagestyle}[1]{}
	\listoffigures
}
\clearpage
\pagestyle{plain}

\mainmatter 
\pagestyle{headings}

\pagenumbering{arabic}
% ----------------------------------------------------------------------------
% Copyright (c) 2016 by Burkhardt Renz. All rights reserved.
% Die Vorlage für eine Abschlussarbeit in der Informatik am Fachbereich
% MNI der THM ist lizenziert unter einer Creative Commons
% Namensnennung-Nicht kommerziell 4.0 International Lizenz.
%
% Id:$
% ----------------------------------------------------------------------------

\chapter{Einleitung}
Oftmals stellt man sich die Frage, mit welchen Werkzeugen eine Aufgabe am effizientesten gelöst werden kann. Insbesondere Informatiker sehen sich mit dem Problem konfrontiert, einer geradezu endlosen, sich stetig wandelnden Auswahl an Programmiersprachen, Konzepten, Algorithmen und dergleichen, gegenüberzustehen.\\
Diese Arbeit soll sich mit den bewährten Programmiersprachen C und R beschäftigen. Unter Verwendung dieser Sprachen, werden verschiedene Ressourcen untersucht, die gebraucht werden, um ein komplexes mathematisches Problem zu lösen.

\section{Zielsetzung}
Das Ziel dieser Arbeit soll sein, ein Rechenzeitvergleich der Programmiersprachen C und R durchzuführen. C ist bekannt dafür, eine Hardwarenahe und schnelle Programmiersprache zu sein. Als Hypothese wird aufgestellt, dass C bessere Rechenzeiten aufweist, als R. Als Vergleichsgrundlage soll die Berechnung des Inversen einer großen Matrix dienen.


% ----------------------------------------------------------------------------

% ----------------------------------------------------------------------------
% Copyright (c) 2016 by Burkhardt Renz. All rights reserved.
% Die Vorlage für eine Abschlussarbeit in der Informatik am Fachbereich
% MNI der THM ist lizenziert unter einer Creative Commons
% Namensnennung-Nicht kommerziell 4.0 International Lizenz.
%
% Id:$
% ----------------------------------------------------------------------------

\chapter{Themenüberblick}
\label{chapter:Themenüberblick}

\section{Verteilte Systeme}
\label{section:Verteilte Systeme}

Verteilte Systeme dienen als Anwendungsfeld für die Instrumentalisierungsbibliothek. Auch das in dieser Arbeit vorgestellte Testbett, sowie das verteilte rendering System, in welches Traktor zum Einsatz kommt, sind solche verteilten Systeme. Dafür ist es wichtig, verschiedene Eigenschaften und Konzepte von verteilten Systemen zu definieren. Diese Eigenschaften und Konzepte nehmen einen entscheidenden Faktor bei der Ermittlung, Analyse und Umsetzung von Anforderungen der Instrumentalisierungsbibliothek ein. Dementsprechenend wird zunächt der Begriff des verteilten Systems definiert:

\begin{quote}
	Ein verteiltes System ist eine Kollektion unabhängiger Computer, die den Benutzern als ein Einzelcomputer erscheinen\footpartcite{10.5555/1202502} 
\end{quote}

\subsection{Eigenschaften eines verteilten Systems}
\label{subsection:Eigenschaften eines verteilten Systems}
	Die Definition von van Steen und Tanenbaum geht mit zwei charakteristischen Eigenschaften von verteilten Systemen einher. Die erste Eigenschaft äußert sich darin, dass alle Komponenten eines Systems unabhängig voneinander agieren können. Komponenten werden in diesem Zusammenhang auch Knoten genannt. Knoten sind Hardwarekomponenten, also pysikalische Recheneinheiten. Auch Prozesse innerhalb einer Hardwareeinheit können Knoten sein. Dabei ist es möglich, dass mehrere Knoten auf einer Hardwareheinheit sind.\footpartcite[p. 2]{VanSteen2017}.

	Knoten sind miteinandern über Netzwerke verknüpft. Innerhalb des Netzwerk kommunizieren Knoten mittels Nachrichten miteinander. Beim Ansprechen eines verteilten Systems soll dem Anfragesteller, zum Beispiel ein Client eines Webservers, nicht ersichtlich sein, dass mehrere Komponenten ein Gesamtsystem bilden, welches die Anfrage verarbeitet und entsprechende Vorgänge ausführt. 
\subsection{Beobachtbarkeit von verteilten Systemen}
\label{subsection:Beobachtbarkeit von verteilten Systemen}
	Unter Beobachtbarkeit versteht man die Möglichkeiten der Betrachtung eines System. Die Symptome, die entstehen können, falls unerwünschte oder unvorhergesehene Zustände in dem System entstehen, sind oft schwer nachvollziehbar. Speziell verteilte Systeme erschweren die Reproduktion solcher Symptome, da das komplexe ineinandergreifen der Komponenten die Erfassung der Vorgänge erschweren. 

	Die Beobachtbarkeit von verteilten Systemen kann in erster Linie durch Überwachung diverser Komponenten des Systems ermöglicht werden. Dabei lassen sich diese Komponenten in drei Kategorien unterteilen. Aufzuzählen sind (\lowroman{1}) \textbf{Hardware} auf denen die Knoten angesiedelt sind, (\lowroman{2}) \textbf{Netwerke}, in denen die Kommunikation der einzelnen Knoten stattfindent und die (\lowroman{3}) \textbf{Anwendung}, welches sich auf alle Knoten verteilten kann.

	
	\textbf{Hardware}\space\space\space Aufgrund der vielzahl an Hardwarekomponenten in einem System gilt es, besonders interessante Komponenten, im Kontext der Fehlerfindung beziehungsweise der Performanceanalyse, zu beobachten. Als Hauptkomponenten der meisten Systeme zählen zum Beispiel die \emph{\gls{cpuLabel}}, die \emph{\gls{gpuLabel}} oder auch die verschiedenen Speicher wie zum Beispiel der \emph{\gls{ramLabel}} oder die \emph{\gls{hddLabel}}. Aus der Beobachtung dieser Komponeten gewinnt man allgemeine Informationen über das Gesamtsystem. Diese Informationen können zum Beispiel dazu genutzt werden, die Auslastung einzelner Knoten zu ermitteln und eventuell auf unbalancierte Nutzung der Konten reagieren zu können oder auftretende \emph{bottlenecks}, das heißt stark auffällige performancebeeinträchtigende Komponenten im System, erkennen zu können. Diese könnten beispielsweise langsame HDDs sein, auf die oft zugegriffen wird.
	
	\textbf{Netzwerk}\space\space\space Die Kommunikation innerhalb des Systems wird durch das Netzwerk, also die Verknüpfung der jeweiligen Knoten miteinandern, ermöglicht. Verschiedene Protokolle zur Regelung des Nachrichtenaustauschs kommen dabei zum Einsatz. Besonders nennenswert sind die Protokolle \emph{\gls{tcpLabel}} und \emph{\gls{ipLabel}}. Diese beiden Protokolle dienen als Grundlage für zwei darauf aufbauende Protokolle. Zum einen das \gls{httpLabel} und das Websocket Protocol. Der Anwendungsfall der Protokolle wird im \cref{chapter:Implementierung} genauer betrachtet.
	
	Auch das Netzwerk generiert aussagekräftige Daten über das verteilte System. Die in dem Netzwerk verkehrenden Datenmengen sind dabei zu betrachten. Diese Datenmengen können auf unterschiedliche Weise gemessen und Schlussfolgerungen aus den Ergebnissen gezogen werden. Gemessen wird Netzwerkverkehr beispielsweise anhand der Größe der Datenpakete, der Geschwindigkeit der Datenpakete vom Sender zum Empfänger oder einer Knotenerreichbarkeitsprüfung. Diese Daten für sich können schon informativ sein. Allerdings lassen sich auch weitere Informationen durch Korrelation gewinnen. Bespielsweise könnte in einem Anwendungsfall eine Korrelation zwischen Geschwindigkeitsanomalien und Tageszeit bestehen. Auch denkbar wären Anomalien, die sich in Paketverlusten äußern. Diese könnten zum Beispiel durch erhöhte Beanspruchung eines bestimmten Knotens beziehungsweise einer bestimmten Komponente ausgelöst werden. Durch feststellung von Korrelationen können Maßnahmen durchgeführt werden, die der auftretende Anomalie entgegensteuert oder gar ganz beseitigt.
	
	\textbf{Anwendung}\space\space\space Die Beobachtung der Anwendung ist, im zusammenspiel mit dem Nachrichtenaustausch über das Netzwerk, zentral. Die Beobachtbarkeit der Anwendung sorgt dafür, dass Anwendungsdaten erhoben, ver- und aufgearbeitet und anschließend präsentiert werden. Die Präsentation der Daten hilft den Verantwortlichen Informationen über die internen Vorgänge des Systems zu gewinnen und entsprechend agieren zu können. Es ist zudem möglich gewisse Daten interpretieren zu lassen. Die daraus gewonnenen Informationen können von weiteren Systemen dazu genutzt werden, automatisiert zu steuern. Grundsätzlich kann man auch hier zwischen drei verschiedenen Datenquellen unterscheiden. Diese drei Datenquellen sind:
	
	\begin{itemize}
		\item Metriken\footpartcite{Watson2017} z.B.:
		\begin{itemize}
			\item Systemdaten
			\item Anzahl von Instanzen
			\item Anfrageanzahl
			\item Fehlerrate
		\end{itemize}
		\item Applikationslogs, z.B.:
		\begin{itemize}
			\item Fehler
			\item Warnungen
			\item Applikationsinformationen
		\end{itemize}
		\item Traces, z.B:
		\begin{itemize}
			\item Segmente
			\item Kontext
		\end{itemize}
	\end{itemize}


\subsection{Ordnung von Events}
\label{subsection:Ordnung von Events}
	
Ein Trace ist die Sammlung von Events die im Laufe des Weges durch ein verteiltes System generiert wurden. Die Knoten, die diesen Weg umfassen, generieren Events, indem sie Programmcode ausführen, welcher instrumtalisiert ist. Diese Events sind die kleinsten Einheiten eines Traces und unterliegen einer Kausalordnung. Das \emph{Happend Before Model} nach Lamport beschreibt die Kausalordnung. Die Kausalordnung ist eine strikte partielle Ordnung.\footpartcite{Garg2002}. 

\begin{figure}[!ht]
	\centering
	\begin{subfigure}[t]{.49\linewidth}
		\centering\includegraphics[width=.8\linewidth]{img/Themenueberblick/PartialOrdering_Concurrent.png}
		\caption[Abbildung]{Zeigt Prozesse P1, P2 und P3. Diese generieren jeweils ein Event.}
		\label{fig:Partial_Ordering_Concurrent_explained_1}
	\end{subfigure}
	\begin{subfigure}[t]{.49\linewidth}
		\centering\includegraphics[width=.8\linewidth]{img/Themenueberblick/PartialOrdering_Concurrent_Explained}
		\caption[Abbildung]{Events (1), (2) und (3) finden parallel statt.}
		\label{fig:Partial_Ordering_Concurrent_explained_2}
	\end{subfigure}
\caption[Partielle Ordnung der Events dreier Prozesse]{}
\end{figure} 


Die \cref{fig:Partial_Ordering_Concurrent_explained_1} stellt eine Situation dar, in der drei Prozesse jeweils ein Event erzeugen. Intuitiv würde man interpretieren, dass (1) von P1 vor (2) von P2 erzeugt wurde und das darauf (3) von P3 folgt. Dies mag stimmen, in der Annahme, das es nur eine globale Zeit als richtwert gäbe. Allerdings lässt sich die Ordnung, in dem Kontext von verteilten Systemen und Nebenläufigkeit, nicht ohne Berücksichtignug verschiedener Einflussfaktoren feststellen. Lamport erläutert, dass in einer Umgebung, in der eine Ordnung anhand eines Zeitstempels physikalischer Zeit festgelegt wird, eine physikalische Uhr vorhanden sein muss.\footpartcite[S. 559]{lamport78} Die Einbindung einer physikalischen Uhr in ein verteiltes System ist eine aufwendige und komplexe Aufgabe. Probleme wie Synchronisation verschiedener Uhren mit keiner absoluten Präzision und dem dazugehörigen Drift, dem algorithmisch entgegengewirkt werden muss, treten auf. \cref{fig:Partial_Ordering_Concurrent_explained_2} zeigt, dass die Events, unter der Berücksichtigung der Definition von Lamport, nebenläufig stattfindet.  Aus diesem Grund ist es naheliegend zu versuche eine Lösung zu finden, die auf physikalische Uhren verzichtet.


Drei Bedingungen müssen nach Lamport erfüllt sein, damit eine Beziehung zwischen Events partiell geordnet ist. 
\begin{quote}
	\glqq (1) Wenn \emph{a} und \emph{b} Events im selben Prozess sind und \emph{a} vor \emph{b} stattfindet, dann \emph{a} $\rightarrow$ \emph{b}. (2) Wenn \emph{a} das Senden einer Nachricht eines Prozesses ist und \emph{b} das Empfangen einer Nachricht eines anderen Prozesses ist, dann \emph{a} $\rightarrow$ \emph{b}. (3) Wenn \emph{a} $\rightarrow$ \emph{b} und \emph{b} $\rightarrow$ \emph{c}, dann \emph{a} $\rightarrow$ \emph{c}.\grqq \:\footpartcite[S. 559]{lamport78}
\end{quote}

\begin{figure}[!ht]
	\centering
	\includegraphics[scale=0.5]{img/Themenueberblick/PartialOrdering_ChildO.png}
	\caption[Partielle Ordnung von nebenläufigen Events]{Zeigt Prozesse P1 und P2. Diese generieren jeweils Events. Logische Reihenfolge des Auftretens der Events anhand der gestrichelten Linien [1], [2], [3] und [4] ersichtlich.}
	\label{fig:Partial_Ordering_Concurrent}
\end{figure}

Um die Feststellung von Lamport zu verdeutlichen wird \cref{fig:Partial_Ordering_Concurrent} untersucht. Die Abbildung stellt zwei Prozesse dar, dabei wird angenommen, dass diese in unterschiedlichen Rechensystemen angesiedelt sind. Das bedeutet, dass auf eine globale physikalische Uhr verzichtet wird. Diese Prozesse können sich mittels senden einer Nachricht und empfangen einer Nachricht verständigen. Das verteilte System generiert insgesamt fünf Events. P1 erzeugt (1) in der logischen Zeitspanne [1] und sendet anschließend eine Nachricht an P2. Das Empfangen wird durch (2) dargestellt. Daraus folgt, dass eine \emph{Happens Before Relation} zwischen (1) und (2) besteht, also (1) $\rightarrow$ (2). Nebenläufig dazu, findet in P1 Event (4) statt. (4) kann durch P2, in dieser Zeitspanne, in keiner Weise beeinflusst werden. Das heißt, dass (4) ein von (2)  kausal unabhängiges Ereignis ist, (2) $\not\rightarrow$ (4). (4) ist aber kausal von (1) abhängig, da es das direkt folgende Event im gleichen Prozess ist, also (1) $\rightarrow$ (4). (4) bildet dabei, als neues abhängiges Event, Zeitspanne [2]. Weitergehend zu folgern ist, (2) $\rightarrow$ (3), weshalb [3] entsteht. Auch hier ist eine kausale unabhängigkeit von (3) zu (4) zu sehen. An dieser Stelle spielt eine weitere Bedingung eine Rolle. Es muss die \emph{Clock Condition} berücksichtigt werden. Diese Bedingungen besagt:

\[
\text{Clock Condition}: \; \text{wenn} \; \emph{a} \rightarrow \emph{b} \; \text{dann} \; C(\emph{a})<C(\emph{b})
\]
Diese Bedingung führt $C$ als logische Uhr ein. Eine logische Uhr ist ein von der physikalischen Zeit unabhängiger Zähler, der einem Event eine Zahl zuweist. Die Clock Conditions ist erfüllt, sobald (1) darauf geachtet wird, dass zwischen zwei Events eines Prozesses die logische Uhr voranschreitet und (2) das einem Event ein Zeitstempel zugewiesen wird, der folgende Eigenschaften hat\footpartcite[S. 560]{lamport78}:
\[
 T_m = C_i \langle a \rangle \; 
 \text{dann} \; 
 {{C}_{j+1}} = \max[C_j,T_m]
\]
Dabei ist $T_m$ der Zeitstempel eines Events, welcher eine Nachricht $m$ zu einem anderen Prozess sendet, zum Zeitpunkt $C_i \langle a \rangle$. Beim empfangenen Prozess wird das erzeugte Event mit dem Zeitstempel ${{C}_{j+1}} = \max[C_j,T_m]$ versehen. Das bedeutet für die Darstellung, dass (3) $\not\rightarrow$ (4), (3) $\rightarrow$ (5) und $C_{P1} \langle 4 \rangle < C_{P2} \langle 3 \rangle$. Diese Annahme kann getroffen werden, weil (2)$\rightarrow$ (3) und somit diese beiden Events nicht zu einem Zeitpunkt mit (4) stattfinden können, weil es $C_{P2} \langle 2 \rangle < C_{P2} \langle 3 \rangle$ widersprechen würde.

 Die \textbf{Transitivität} ist durch (1) $\rightarrow$ (2) und (2) $\rightarrow$ (3) gegeben. Das heißt eine Relation zwischen (1) und (3) besteht, also (1) $\rightarrow$ (3). Die zweite Charakteristik, dass ein Event nicht vor sich selbst stattfinden kann, ist gegeben.(1) $\rightarrow$ (1) kann also nicht bestehen und ist somit \textbf{irreflexiv}. Zuletzt die ist durch die Tatsache, dass ein Event nicht vor und nach einem anderen parallel bestehen kann, also (1) $\rightarrow$ (2) und (2) $\rightarrow$ (1), gezeigt, dass die Relation \textbf{antisymmetrisch} ist. Die Kombination der drei Charakterisitken machen eine strikte partielle Ordnung aus, eine sogenannte Kausalordnung.

Die gezeigte Kausalordnung von Events stellt das Fundament zur Erhebung von Tracedaten in verteilten Systemen dar und ist somit unerlässlich, um ein verteiltes System beobachtbar zu machen, also einen Teil der \emph{Oberservability} zu ermöglichen.

\section{Distributed Tracing}
\label{subsection:Erkenntnisinteresse}
Der Begriff des Verteilte Systems wurden bereits nach Tanenbaum definiert. Allerdings lohnt es sich eine alternative Definition zu betrachten, die eine andere Perspektive ermöglicht. 

\begin{quote}
	Ein verteiltes System ist ein System, mit dem ich nicht arbeiten kann, weil irgendein Rechner abgestürzt ist, von dem ich nicht einmal weiß, dass es ihn überhaupt gibt.\footpartcite{lamport87}
\end{quote}


Diese Definition lässt sich so auffassen, dass Lamport im Jahr 1987 die Problematik der Fehlersuche während der Laufzeit, des Debuggings während der Entwicklungsphase und des organisatorischen Aufwands im Allgemeinen, also damit der grundsätzlich hohen Unübersichtlichkeit und Komplexität von verteilten Systemen, beschreibt. Diese Probleme und Herausforderungen, mit denen man in der heutigen Zeit der serviceorientierten Architektur beziehungsweise der Microservicearchitektur konfrontiert wird, können durch distributed tracing angegangen werden. So kann zum Beispiel Fehlerquellenermittlung beschleunigt werden. Tracing ist ein Konzept, welches als Werkzeug umgesetzt werden kann. Durch das Erheben von Tracedaten lassen sich detailierte Schlussfolgerungen über einzelne \emph{Requests} und ihren Weg durch das verteilte System schließen. Dies verlangt allerdings das direkte Eingreifen in den Quellcode. Der alternative Ansatz des \emph{Blackbox Monitoring}, also das Überwachen eines Systems mit eingeschränkten Informationen, wird in sich überschneidenden Anwendungsbereichen eingesetzt. Das Blackbox Monitoring versucht das System von Außen zu betrachten. Dabei wird keine Instrumentalisierung von Quellcode vorgenommen. Dieser sammelt Daten aus bestehenden Logs und Netzwerkschnittstellenüberwachung. Der Ansatz des Blackboxmonitoring ist sinnvoll, wenn mit vielen Softwarekomponenten gearbeitet wird, auf deren Entwicklung man keinen direkten Einfluss hat. Dazu zählen beispielsweise proparitäre Software, ohne Zugang zu dem Quellcode. Allerdings ist die Rekonstruktion der Requestpfade unzuverlässig und die Fehlerfreiheit, der gesamten Kausalordnung aller erzeugten Events, aufgrund der eingeschränkten Informationen, nicht ohne Nachteile, gegeben. 


Distributed tracing umfasst zwei Teilbereiche der im \cref{subsection:Beobachtbarkeit von verteilten Systemen} beschriebenen Beobachtbarkeit von verteilten Systemen. Das ist zum einen die verteilte Anwendung mit ihren einzelnen Serviceskomponenten und zum anderen das Netzwerk über das Nachrichten ausgetauscht werden. In \cref{chapter:Problembeschreibung} wird anhand eines minimalbeispiels beschrieben, welche Rollen diese beiden Aspekte in der Generierung und Ordnung von Events zu spielen haben.

Ziel von distributed tracing soll sein, mit möglichst wenig \emph{overhead} und \emph{minimalem Aufwand} tiefgreifenden Einblick in eine verteilte Anwendung zu gewinnen. Mit wenig overhead ist gemeint, dass das System nicht zu stark beeinträchtigt wird. Das heißt, dass es zu keiner unvereantwortlichen Einbüßung von Leistung kommen darf. Das Bedürfnis nach minimalem Aufwand stammt von der Natur der Microservice-Architektur. Das Prinzip von loose gekoppelten und jederzeit austauschbaren Microservices fordert eine schnelle und einheitliche  Möglichkeit, Einblick in die Komplexität von verteilten Systemen zu gewinnnen. Dabei ist es erforderlich und aus Entwicklersicht verständlich, dass der Instrumentalisierungsanteil des Services nur ein kleinen Teil der Gesamtlogik ausmachen darf. Den die Entwickler wollen sich auf die Businesslogik konzentrieren und distributed tracing soll den Entwicklungsprozess unterstützen und nicht durch übermäßige Investitionsanforderungen beeinträchtigen.

 
\section{Entwicklung einer Tracingbibliothek}
\label{subsection:Entwicklung einer Tracingbibliothek}

Das Konzept von \emph{distributed tracing} ist als Bibliothek in der Programmiersprache C\# umgesetzt. Drei Konzepte sind dabei zu definieren. Diese sind \textbf{Profiling}, \textbf{Tracing}, und \textbf{Instrumentalisierung}.  

	\textbf{Profiling} ist der Prozess des Analysierens einer Anwendung durch Erhebung von anwendungsbezogenen Daten. Dabei kommen Profilingwerkzeuge wie z.B. \emph{perf} zum einsatz, die Profilingdaten erheben. Diese Daten bestehen unter anderem aus aufgerufenen Funktionen, CPU Auslastung und Laufzeiten von Funktionen. Die Events, die durch Profiling erhoben werden, weisen keine Relationen zueinander auf. 
	
	\textbf{Tracing} ist eine Konzept zur Anwendungsüberwachung. Dabei werden Events in relation zueinader gesetzt, die die chronologische Reihenfolge der Events repräsentieren.
	
	\textbf{Instrumentalisierung} ist das Implementieren von Anwendungslogik, die dafür sorgt, das Daten erhoben werden können, die zur Performanceanalyse und zur Fehlerdiagnose dienen. Die Daten werden z.B. von Tracingwerkzeugen genutzt.

Die Bibliothek wird als Tracingwerkzeug konzipiert, die sich Instrumentalisierung zu nutzen macht, um ihre Daten zu erheben.
Die Bilbiothek über den Packetmanager \emph{Nuget} veröffentlicht. Dadurch soll minimaler Integrationsaufwand in Systeme entstehen. Dies ermöglicht in den Entwicklungsumgebungen wie zum Beispiel \emph{Unity} auf Linux, wie auch auf Windows, eine einfache und schnelle integration. Innerhalb des Entwicklungsprozess ist eine \gls{ciGlossar} Pipeline aufgebaut worden, die dafür sorgt, dass die Bibliothek über Nuget verfügbar gemacht wird.

% ----------------------------------------------------------------------------

% ----------------------------------------------------------------------------
% Copyright (c) 2016 by Burkhardt Renz. All rights reserved.
% Die Vorlage für eine Abschlussarbeit in der Informatik am Fachbereich
% MNI der THM ist lizenziert unter einer Creative Commons
% Namensnennung-Nicht kommerziell 4.0 International Lizenz.
%
% Id:$
% ----------------------------------------------------------------------------

\chapter{Problembeschreibung}
\label{chapter:Problembeschreibung}
In diesem Kapitel wird die Problematik, um das Generieren und Ordnen von Events in einem verteilten System mit asynchroner Kommunikation, beschrieben. Dabei betrachten man die Relevanz der Problemstellung. Ausserdem werden Fragen aufgestellt, die diese Problemstellung umfassen. Anschließend wird eine Anforderungsanalyse durchgeführt.

\section{Anwendungsüberwachung}
\label{section:Tracing von Anwendungen}
Viele Bereiche der Wirtschschaft, der Wisschenschaft und grundsätzlich des alltäglichen Lebens sind Software unterstützt. Trends wie beispielsweise \gls{iotLabel}, Hausautomatisierung, Mobile Geräte, etc. sind Anwendungsbeispiele. Diese sind aus ihrer Natur heraus stark verteilte Anwendungen. Aber auch potentiell neue Anwendungsbereiche, wie zum Beispiel verteiltes \gls{renderingGlossar}, benötigen detailierte Einsicht in die internen Vorgänge der Anwendung. 

Dabei spielen zwei Eigenschaften in der Überwachung der Anwendung eine zentrale Rolle. 
Zum einen ist das die (\lowroman{1}) \emph{Performance} und zum anderen die (\lowroman{2}) \emph{Korrektheit}.


\textbf{Performance} \space\space\space 
Viele Anwendungsbereiche setzten gewisse Rahmenbedingungen, die erfüllt werden müssen. 
Nutzererwartungen im Bezug auf interaktive Systeme, welches einer der beiden Anwendungsfälle der Instrumentalisierungsbibliothek ist, äußern sich beispielsweise in der Reaktionszeit der Anwendung auf Benutzereingaben. 
Das Rendering nimmt dabei nur einen Teil der Gesamtlatenz ein.
Ein beispielhafter Gesamtpfad, der durch die verteilte Renderinganwendung genommen werden kann, besteht aus dem Senden der Benutzereingabe, der Übermittlung der Benutzereingabe zum verteilten System, der Verarbeitung der Eingabe und der Übermittlung des Ergebnisses an die Benutzeranwendung.
\cref{fig:Anwendungsueberwachung_Gesamtsystem} verdeutlicht diesen Pfad von kausal relatierenden Events.
Dabei ist jede Komponente des Pfades ein generiertes Event.
Zu sehen ist, dass das (1) Senden der Benutzereingabe vor dem (2) Übermitteln der Benutzereingabe stattfinden.
Anschließend wird die Eingabe (3) Empfangen.
Auch die (4) Verarbeitung im verteilten System, das für den Benutzer, wie in \cref{subsection:Eigenschaften eines verteilten Systems} definiert, nicht als solches kenntlich sein muss, generiert in diesem Beispiel ein Event.
Die Antwort wird (5) gesendet und die (6) Übermittlung wird durchgeführt.
Zuletzt wird die Antwort (7) empfangen. Das Empfangen schließt den Pfad ab. Die Gesamtdauer des Pfades wird als Latenz eines Frames bezeichnet.
Daraus kann eine Durschschnittslatenz über eine Zeitspanne berechnet werden, welches als Performanceindikator dient. Die Zeitspanne zwischen den einzelnen Events können verglichen werden. Dabei ist es möglich sog. \emph{Bottlenecks} zu identifizieren. Bottlnecks sind Vorgänge, die einen großteil der Gesamtdauer ausmachen. Sind werden durch die Zeitspanne zwischen zwei Events, die auf dem \emph{kritischen Pfad} liegen, bestimmt. Diese Art der Anwendungsüberwachung soll die Möglichkeit bieten, Bottlenecks zu identifizieren. Wie in \cref{subsection:Ordnung von Events} beschrieben, verlangt eine Messung der Zeit über verschiedene physikalisch Entitäten entweder eine globale physikalische Uhr oder jeweils eine physikalische Uhr in jeder Entität, die über alle Entitäten synchrone sind beziehungsweise, synchronisiert werden. Dabei stellt sich die Frage, \emph{inwiefern eine solche Zeitmessung von Zeitspannen zwischen Events konzipiert werden kann. } 


\begin{figure}[!ht]
	\centering
	\includegraphics[scale=0.5]{img/Problembeschreibung/Anwendungsueberwachung_Gesamtsystem.png}
	\caption[Kausaler Pfad einer Vorgangs in dem verteilten rendering System]{Kausaler Pfad einer Vorgangs in dem verteilten rendering System}
	\label{fig:Anwendungsueberwachung_Gesamtsystem}
\end{figure}

\textbf{Korrektheit} \space\space\space Die Korrektheit eines Systems ist dann gegeben, wenn die Eigenschaften eines Systems einer \emph{Spezifikation} entsprechen. Das bedeutet, dass die Beobachtung des Verhaltens einer (verteilten) Anwendung nicht ausreicht, um seine Korrektheit zu beweisen. Tracing soll also nicht die Korrektheit einer verteilten Anwendung beweisen. Tracing kann aber dabei unterstützen, indem es das Verhalten einer Anwendung beobachtbar macht. Insbesondere die Zusammenhänge der Komponenten und die entstehenden Nebenläufigkeiten sind erschwerende Faktoren in der Verifikation. So stellt sich die Frage, ob \emph{kausal zusammenhängende Events derart dargestellt werden können, dass anhand einer Visualisierung feststellbar ist, ob das Verhalten der Anwendung starke Ausreißer, die auf Fehlimplmenentierung deuten könnten, aufweist.}


\section{Zusammenführung von Events}
\label{section: Ordnungsproblem von Events}
\begin{figure}[!ht]
	\centering
	\includegraphics[scale=0.5]{img/Problembeschreibung/distributed_system_application_minimal.png}
	\caption[Minimale Struktur eines verteilten Systems]{Minimale Struktur eines verteilten Systems, bestehend aus zwei Komponenten}
	\label{fig:distributed_system_application_minimal}
\end{figure}
\begin{figure}[!ht]
	\centering
	\begin{subfigure}[t]{.49\linewidth}
		\centering\includegraphics[width=0.9\linewidth]{img/Problembeschreibung/distributed_system_application_inside.png}
		\caption[Abbildung]{zeigt Knoten mit instrumentalisiertem Anwendungscode}
		\label{fig:distributed_system_application_inside}
	\end{subfigure}
	\begin{subfigure}[t]{.49\linewidth}
		\centering\includegraphics[width=\linewidth]{img/Problembeschreibung/distributed_system_network.png}
		\caption[Abbildung]{Netzwerkkommunikation über TCP/IP}
		\label{fig:distributed_system_network}
	\end{subfigure}
	\caption[Anwendungsinstrumentalisierung und Netzwerkkommunikation über TCP/IP in verteilten Systemen]{}
\end{figure} 

Man definieren ein minimales verteiltes System, welches das verteilte rendering System vereinfacht darstellt. Dieses besteht aus zwei Komponenten. \cref{fig:distributed_system_application_minimal} bildet ein solches System ab. Die Knoten beinhalten zwei für das Generieren und Ordnen von Events interessante Aspekte. Dies ist zum einen die in \cref{fig:distributed_system_application_inside} dargestellte verteilte Anwendung mit ihrem instrumentalisiertem Code und zum anderen das in \cref{fig:distributed_system_network} dargestellte Netzwerk, über welches Nachrichten ausgetauscht werden. 

 \begin{figure}[!ht]
	\centering
	\includegraphics[scale=0.5]{img/Problembeschreibung/problembeschreibung_flamengraph.png}
	\caption[Visualisierung von CPU Performancedaten]{Visualisierung von CPU Performancedaten dargestellt als Flammengraph von Brendan D. Gregg \cite{BrendanGregg2011}}
	\label{fig:problembeschreibung_flamengraph}
\end{figure}

Die Komponenten des Systems besitzen jeweils einen Linux Kernel. Der Kernel bietet die Funktionalität \emph{perf\_events} zu erheben. 
\begin{quote}
	perf\_events ist ein eventorientiertes Überwachungswerkzeug, welches helfen kann, Leistung zu verbessern und Fehelerquellen von Funktionen zu lokalisieren. 
	\footpartcite{BrendanGreggPerf}
\end{quote}
Von dem Szenario ausgehend, dass man Events innerhalb des Minimalbeispiels analysieren möchte, eignen sich Flammengraphen.
 Der in \cref{fig:problembeschreibung_flamengraph} gezeigte Flammengraphe stellt perf\_events dar, die während einer TCP Kommunikation erhoben worden sind.  Dabei ist die Länge der Balken, die Zeit, die das Event, relativ zur Gesamtzeit der Messung, insgesamt eingenommen hat. Die Profilingdaten weisen keine kausalität auf. Die erhobenen perf\_event Daten sind Stichproben. Man geht in diesem Beispiel allerdings davon aus, das alle Events aufgenommen worden sind. Diese Darstellung erlaubt es, die Events \emph{eines} Systems genau zu beschreiben. Nun kommt das zweite System hinzu, mit dem die Kommunikation stattgefunden hat. Auch hier sei gegeben, dass das zweite System Daten generiert hat, welche zu einer ähnlich aufgebaute Visualisierung führt. Die beiden Flammengraphen werden miteinandener verbunden. Dies führt zu einer dreidimensionalen Darstellung von Flammengraphen, gezeigt in \cref{fig:flamegraph_3D}. Wie in einer TCP-Verbindung üblich, wird eine Kommunikationskanal aufgebaut. Über den Kanal können Nachrichten ausgetauscht werden. Anschließend wird die Verbindung mit einem Vier-Wege-Handschlag beendet. Die obersten Blöcke und ihre systemübergreifenden Verbindungslinien, dargetellt durch die gestrichelten Linien mit Pfeilrichtung, stellen die Terminierung der TCP-Verbindung dar.
 
 Der Terminierungsprozess wird genauer betrachtet. \cref{fig:flamegraph_3D_closing} zeigt vier Events. \textbf{A} ist die \emph{FIN} Markierung des Initiators. Sie leitet die Terminierung ein. \textbf{C} stellt das Empfangen und Beantworten mittels \emph{ACK} und \emph{FIN} dar. \\
 \textbf{B} ist der Terminierungsmoment des Initatorsystems. Dieser findet nach dem Zeitpunkt des Eintreffens von \emph{FIN} des Empfänger statt. Dieser Zeitpunkt ist das Senden des letzten \emph{ACK} des Initators, addiert mit einer Konstante \emph{Timeout}.  Event \textbf{B} ist also definiert als:
 
\[
	\text{B}: \; Ack_{init} + Timeout  
\]

 \textbf{D} ist der Terminierungsmoment des Empfängersystems. Dieser Zeitpunkt ist das Erhalten der letzten, vom Initiatorsystem gesendeten, \emph{ACK} Makierung. Die unbekannte Variable \emph{Übertragungszeit} nimmt Einfluss auf den Zeitpunkt. Event \textbf{D} ist also definiert als:
 \[
 \text{D}: \; Ack_{init} + Übertragungszeit 
 \]
 
 Zu untersuchen sind die Relationen zwischen diesen vier Events.
 Dabei sind zwei Relationen, wie in \cref{subsection:Ordnung von Events} beschrieben, als $\text{A}\rightarrow\text{B}$ und $\text{C}\rightarrow\text{D}$ definiert. Durch die kausale Abhängigkeit von $\text{C}$ von $\text{A}$ gilt $\text{A}\rightarrow\text{C}$. Durch die \emph{Transitivität} ist entsprechend  $\text{A}\rightarrow\text{D}$ gegeben.  Es ist zu untersuchen, ob $\text{B}\rightarrow\text{D}$ gilt.
 Dabei sind die Bedingungen, die von Lamport definiert worden sind, zu betrachten. Da zwei Systeme miteinander kommunizieren, muss folgende Bedingung erfüllt sein, sodass eine \emph{Happens-Before} Relation gegeben ist. 
 \begin{quote}
 	(\lowroman{1}) Wenn $\text{a}$ das Senden einer Nachricht ist und $\text{b}$ das Empfangen derselben Nachricht in einem anderen System ist, dann $\text{a}\rightarrow\text{b}$\footpartcite{lamport78}
 \end{quote}

Nach der Definition von \textbf{D} ist es mit $\text{b}$ gleichzusetzen, somit ist ein Teil der Bedingung erfüllt. \textbf{B} ist jedoch nicht das Senden der Nachricht, also des letzten \emph{Ack}, sonder ein Event, welches darauf folgt. Die Events sind nebenläufig. 

%Für die TCP-Kommunikation ist dieser spezielle Zeitpunkt $\text{B}$ nicht relevant. 
%Die Zustandsdefinitionen des TCP erlauben eine fehlerfreie Terminierung. 
%Allerdings könnte dieser Zeitpunkt ein Anwendungsfall für Tracing sein und eine eigene Terminierung für eine ähnliche Situation benötigen. 
%Es ist zu untersuchen, ob Events, ähnlich wie TCP Pakete, die über mehrere Tracer verteilt sind, durch einen Terminierungsprozess eines Traces allesamt erfasst werden können.

Aus dieser Darstellung folgern zwei Fragestellungen:
\begin{quote}
	(\lowroman{1}) Müssen Traceingwerkzeuge ähnliche Terminierungsmechanismen implementieren, wie z.B Netzwerkprotokolle, um alle Events, in allen Prozessen, zu erkennen?
	
	(\lowroman{2}) Welche Darstellungensformen gibt es für verteilte Tracingdaten?
\end{quote}



\begin{figure}[!ht]
	\centering
	\includegraphics[scale=0.5]{img/Problembeschreibung/flamegraph_3D_closing.png}
	\caption[Flammengraph TCP schließung]{Detailierte Betrachtung des in \cref{fig:flamegraph_3D} gezeigte Nachrichtenaustauschs. Stellt die Schließung einer TCP Verbindung dar}
	\label{fig:flamegraph_3D_closing}
\end{figure}

\section{Anforderungsanalyse}
\label{section:Anforderungsanalyse}

Das Systems, für das die Instrumentalisierungsbilbiothek entwickelt wird, ist ein System für verteiltes Rendering. Die Instrumentalisierungsbilbiothek muss notwendige Funktionalitäten spezifizieren, die es ermöglichen ein Modell aus kausal abhängigen Events darzustellen. Zur Erstellung des Modells muss sich mit den Funktionalitäten der \textbf{Eventgenerierung}, der \textbf{Eventrelation}, der \textbf{Synchronisation von Eventgeneratoren}, der \textbf{Eventübermittlnug} und der \textbf{Ordnung von Events} beschäftigt werden. Im Fokus der Interpretation des Modells soll die \textbf{end-zu-end Latenz}, sowie die \textbf{Generierungszeit eines Frames} stehen. Rahmenbedingungen wie die eingeschränkte \textbf{Nachrichtenmodifikation} sind zu berücksichtigen.

Semantisch relevante Ereignisse sind zu definieren. Eine Funktionalität muss geschaffen werden, die es erlaubt, diese Ereginisse als ein Event abzubilden. Die Generierungfunktionalität muss dafür sorgen, dass die Events einem spezifizierten Aufbau aufweisen, um weiterverarbeitet und ausgewertet werden zu können. Die Nutzung einer standarisierten und erprobten \gls{apiLabel} ist wünschenswert.
 
\subsection{Funktionalitäten}

\subsubsection{Eventgenerierung}
\label{subsubsection:Eventgenerierung}
Events müssen in einem für die Anwendung semantisch relevanten Bereich generiert werden können. Die Generierung eines Event, welches den Startpunkt eines Traces darstellt, muss dafür sorgen, dass der Traces eindeutig identifizierbar ist. Alle Events, die auf ein anderes Event folgen, müssen eine Relation auf das Elternevent aufweisen.

\subsubsection{Eventrelation}
\label{subsubsection:Eventkorrelation}
Es muss ein Modell für Events konzipiert werden. Das Modell muss in der Lage sein, Relationen abbilden zu können. Diese Relationen sollen die kausalen Zusammenhänge der Events darstellen. 

\subsubsection{Synchronistion von Eventgeneratoren}
\label{subsubsection:Synchronistion von Eventgeneratoren}
Eventgeneratoren sind oftmals auf verschiedenen Komponenten des verteilten Systems angesiedelt. Wie kann also ein Konzept aussehen, dass dafür sorgt, dass Events geordnet werden können.

\subsubsection{Eventübermittlung}
\label{subsubsection:Eventübermittlung}
Damit ein Kausalpfad erstellt werden kann, müssen die Events, die in einem anderen System generiert werden und zum kausalpfad gehören, in einer Form zusammengeführt werden können.

\subsubsection{Eventkontext}
\label{subsubsection:Eventkontext}
Das erste generiert Event bildet den Startpunkt eines neuen Traces dar. Mehrere Anfragen können zur gleichen von dem in \cref{fig:distributed_system_application_minimal} beschriebenen System angenommen werden. Dadurch ist gegeben, dass mehrere Traces parallel existieren. Um diese eindeutig zu machen muss ein Tracekontext definiert werden, der es erlaubt, Traces eindeutig, über Prozessgrenzen hinaus, zu identifizieren und Events diesem zuzuordnen.

\subsection{Rahmenbedingen}
\label{subsection:Rahmenbedingen}
\subsubsection{end-zu-end Latenz}
\label{subsubsection:end-zu-end Latenz}
Die durchschnittliche Antwortzeit des verteilten rendering Systems ist im Durchschnitt 100ms. Die durchschnittliche Generierungszeit eines Frames beträgt 16ms. Die Instrumentalisierung des Quellcodes fügt zusätzliche Programmlogik hinzu, weshalb eine erhöhte Verarbeitungszeit entsteht. Diese Verarbeitungszeit soll das verteilte rendering System möglichst geringfügig belasten. Vorallem die Generierungszeit der Frames muss unverändert sein.

\subsubsection{Generierungszeit eines Frames}
\label{subsubsection:Generierungszeit eines Frames}
Die Renderinggeschwindigkeit wird anhand der Zeit gemessen, also wieviele \emph{ms} gebraucht werden, um ein Bild zu generieren. gemessen Der Generierungsprozess eines \emph{Frames} umfasst vier Ebenen. Diese Ebenen sind die Applikationsebene, die Geometrieprozessierung, die Rasterung und die Pixelprozessierung. Die Verarbeitung wird, abhängig von der bearbeitet Ebene, von der \gls{cpuLabel} oder der \gls{gpuLabel} durchgeführt. Es ist wünschenswert GPU und CPU Aktivitäten überwachen zu können.

\subsubsection{Nachrichtenmodifikation}
\label{subsubsection:Nachrichtenmodifikation}

Die Generierung von Events kann von zwei Perspektiven aus betrachtet werden.
Zum einen die \emph{Blackbox} Perspektive und zum anderen die \emph{Whitebox} Perspektive.

Bei dem Blackboxansatz, wird die Generierung angestoßen, sobald Schnittstellen angesprochen werden. Dabei werden betriebssystemspezifische Funktionalitäten genutzt, um diese Betriebssystemereignisse zu erkennen. Diese Ereginisse können erkannt, aufgearbeitet und als Events gespeichert werden. Betriebssystemspezifische Ereignisse sind vorallem ausgehende und eingehende Nachrichten, die von den Netzwerkschnittstellen verarbeitet werden. \cref{fig:distributed_system_network} zeigt eine auf dem TCP/IP Stack basierende Nachricht. Die Daten der Senderadresse, der Empfängeradresse und einem Zeitstempel könnten genutzt werden. Allerdings ist das Fehlen von Applikationsinformationen ein entscheidendes Problem. Das Ziel des Blackboxansatz ist die minimale Vorraussetzung von \emph{a priori} Informationenen über Kommunikationswege, über den Aufbau von Applikationsnachrichten, die Semantik der Anwendung und den Aufbau des verteilten Systems.\footpartcite{Aguilera2003}. Allerdings sind diese Daten äußert wichtig, um ein tiefgreifendes Verstädnis des verteilten Systems zu gewinnen. 

Der Whiteboxansatz nutzt Instrumentalisierung des Quellcodes, um die Eventgeneriernug anzustoßen. Dabei wird vorausgesetzt, dass die Semantik der Anwendung, Informationen über den Aufbau von Nachrichten, den Aufbau des Systems und die Kommunikationswege zwischen Komponenten bekannt sind. Bei der Notwendigkeit einer Modifizierung von Nachrichten weist dieser Ansatz jedoch auch schwächen auf. Im Anwendungsfall des verteilten rendering Systems fehlt die Möglichkeit, Nachrichten, innerhalb der Anwendung, um Tracingdaten zu erweitern.


% ----------------------------------------------------------------------------

% ----------------------------------------------------------------------------
% Copyright (c) 2016 by Burkhardt Renz. All rights reserved.
% Die Vorlage für eine Abschlussarbeit in der Informatik am Fachbereich
% MNI der THM ist lizenziert unter einer Creative Commons
% Namensnennung-Nicht kommerziell 4.0 International Lizenz.
%
% Id:$
% ----------------------------------------------------------------------------

\chapter{Design}

\section{Datenmodell}
\subsection{Eventmodell}
\subsection{Eventgraph}
\section{Verarbeitungsmodell}
\subsection{Agenten}
\subsection{Collectoren}

% ----------------------------------------------------------------------------

% ----------------------------------------------------------------------------
% Copyright (c) 2016 by Burkhardt Renz. All rights reserved.
% Die Vorlage für eine Abschlussarbeit in der Informatik am Fachbereich
% MNI der THM ist lizenziert unter einer Creative Commons
% Namensnennung-Nicht kommerziell 4.0 International Lizenz.
%
% Id:$
% ----------------------------------------------------------------------------
\chapter{Implementierung}
\label{chapter:Implementierung}

In diesem Kapitel werden Implementierungen von Prototypen gezeigt, die im Kapitel \cref{chapter:Design} besprochen sind. In \cref{section:Instrumentalisierungsbibliothek: Traktor} werden die implementierungsdetails der Instrumentalisierungsbilbiothek \textbf{Traktor} vorgestellt. In \cref{section:Traktor Agent} und in \cref{section:Traktor Registry} werden die Traktor Services beschrieben. Anschließend werden zwei Anwendungsfälle gezeigt. Der Anwendungsfall des rendering System in \cref{section:Unity Rendering System} und der Anwendungsfall einer verteilten Webanwendung in \cref{section:Webserver Entwicklungsumgebung}

\section{Instrumentalisierungsbibliothek: Traktor}
\label{section:Instrumentalisierungsbibliothek: Traktor}
In diesem Kapitel wird die Implementierung der Instrumentalisierungsbibliothek Traktor vorgestellt. Es wird das Datenmodell umgesetzt, welches in \cref{section:Datenmodell} konzipiert ist und gezeigt inwiefern Traktor die Anforderungen erfüllt.

Traktor ist eine Instrumentalisierungsbibliothek, die sich die OpenTracing API als Vorbild nimmt. Die OpenTracing API ist eine herstellerneutrale instrumentalisierungs API, die unter der Schirmherschafft der \gls{cncfLabel} steht, welches Teil der gemeinnützig agierenden \gls{lfLabel} ist.

Die zwei zentralen Einheiten der Traktor Instrumentalisierungsbibliothek sind der in \cref{subsection:Tracer} gezeigte \textbf{Tracer} und der in \cref{subsection:Span} \textbf{Spans}. Weitere Klassen, die den in \cref{subsection:Tracingcontext innerhalb eines Systems} beschriebenen Designansatz implementieren, sind der in \cref{subsection:Scope} vorgestellte Scope und der in \cref{subsection:ScopeManager} vorgestellte ScopeManager. Die Kontextpropagierung, die mit einer Websocketverbindung innerhalb des Tracer und der, in \cref{section:Traktor Registry} gezeigten, Traktor Registry, umgesetzt ist, implementiert den in \cref{subsection:Tracingcontext über Systemgrenzen} erläuterten Designansatz. Ein Gesamtüberblick mit allen Relationen der Klassen, wird in \cref{subsection:UML-Klassendiagramm der Bibliothek} gegeben.

\subsection{Tracer}
\label{subsection:Tracer}
Der Tracer ist die zentrale Verwaltungseinheit der Instrumentalisierung. Dieser verwaltet die Verbindung der Anwendungsinstrumentalisierung zu dem Agenten und der Registry. Der Tracer kümmert sich um das Generieren von Spans und die Herstellung von Kausalzusammenhängen zwischen den Spans. Ausserdem stellt der Tracer die Funktionalitäten zur Kontextpropagierung über die Registry bereit. Die Spans sind Darstellungen von ausgeführter Arbeit in einer instrumentalisierten Anwendung. 

Die Tracer API erweitert die OpenTracing API um vier Funktionen, die notwendig sind, um sich mit der Tracinginfrastuktur zu verbinden. Die \emph{Configure} Funktion sorgt für einen Verbindungsaufbau, abhängig von den übergebenen Parametern. Ein vollständiger Verbindungsaufbau ist nötig, um eine Kontextpropagierung zu ermögliche. Ausserdem wird der \gls{udpLabel} Klient initzialisiert, damit beendete Spans reportet werden können. Der UDP-Klient hat einen Port zu belegen, der angegeben werden muss. Die IP-Adressen der Services sind dem Tracer bekannt zu geben. Auch der Port auf der die Anwendung lauscht, muss übergeben werden. Die Configure Funktion verwendet die \emph{Register} Funktion zur Erstellung einer \textbf{WebSocketClient} Instanz. Die Websocketverbindung wird über die Lebensdauer des Tracer offen gehalten. Über diesen Kommunikationskanal werden die Kontextinformationen übermittelt. \cref{listing:Tracer Verbindungsaufbau} zeigt eine Beispielhafte Konfigurierung. Dabei wird eine Tracerinstanz instanziiert und mit den Adressen der Traktorservices konfiguriert.

\begin{minipage}[]{\textwidth}
	\begin{lstlisting}[frame=trBL]
	using Traktor;
	
	var registryAddress = "localhost";
	var registryPort = "8080";
	var agentAddress = "localhost"
	var agentPort = 13337;
	var reporterPort = 13338;
	
	Tracer tracer = new Tracer()
	tracer.Configure(registryAddress,registryPort,agentAddress,
				agentPort,reporterPort);
	
	\end{lstlisting}
	\captionof{lstlisting}{Tracer Verbindungsaufbau mit der Registry und dem Reporter }
	\label{listing:Tracer Verbindungsaufbau}
\end{minipage} 

Die Kontextdaten werden in einer \emph{Carrier} Instanz gespeichert und über die Leitung gesendet. In der Implementierung gibt es einen Carriertyp. Der BinaryCarrier ist ein Datencontainer, der die Kontextinformationen als MemoryStream speichert. Der MemoryStream speichert die Daten im Arbeitsspeicher als Bytearray. Das Format würde für die Übertragung über eine TCP/IP Verbindung ausgewählt. 
Die Spangenerierung wird durch eine \emph{SpanBuilder} Instanz geregelt. Der Operationsname wird der SpanBuilder Instanz mitgegeben und für die Generierung des Spans genutzt. Die Kontextpropagierung wird durch die beiden sich ergänzenden asynchronen Funktionen \emph{SendContext} und \emph{ReceiveContext} implementiert. Diese nutzen die \emph{Extract} und \emph{Inject}, um die BinaryCarrier zu erstellen beziehungsweise die Daten aus dem Carrier zu extrahieren.

Der Tracer hält eine \textbf{Scopemanager} Instanz. Der Scopemanager verwaltet die Spans. Über den Scopemanager kann der Tracer den aktuell aktiven Span ermitteln.


\subsection{Span}
\label{subsection:Span}
Der Span ist beinhaltet alle relevanten Daten, die für das Repäsentieren eines Events in einem System notwendig sind. Darunter zählt ein Operationsname. Der Operationsname kann beispielsweise die Sematik des Events beschreiben. Auch bietet sich ein Funktionsname innerhalb der Anwendung an, falls der Span genau diesen umfasst. Der \textbf{Startzeitstempel} und der \textbf{Endzeitstempel} beschreibt die Zeitspanne, in der das Event stattfindet. Diese sind in der Implementierung \emph{Datetime} instanzen. Das Datetimeformat, welches für die tracinginfrastruktur genutzt wird, hat folgenden Aufbau:

\begin{minipage}[]{\textwidth}
	\begin{lstlisting}[frame=trBL]
	MM/dd/yyyy hh:mm:ss.ffff tt
	\end{lstlisting}
	\captionof{lstlisting}{Zeitstempelformat der Spans}
	\label{listing:Tracer Verbindungsaufbau}
\end{minipage} 

Der Aufbau orientiert sich an dem amerikanischen Zeitformat. \textbf{MM} steht für den Monat, \textbf{dd} für den Tag und \textbf{yyyy} für das Jahr. Interessanter wird es bei dem Tageszeitformat, den in einem System bei dem Zeit eine kritische Rolle spielt, sind kleinste Zeitunterschiede von Bedeutung. \textbf{hh} und \textbf{mm} sind entsprechend die Stunden und Minunten. \textbf{ss.ffff} entspricht den Sekunden und dem tausendstel einer Sekunde. Die \textbf{Systemuhr Resolution} ist an dieser Stelle erwähnenswert. Die Resolution einer Uhr beschreibt die kleinste Einheit von Zeit, die akkurat von einer Uhr gemessen werden kann. Die Resolution der Systemuhr hängt von dem Betriebsystem ab. In einem Windows 8  Betriebsystem ist die Standardresolution bei 15.6 ms. Das .NET Framework, welches für die Implementierung genutzt wird, behandelt verschiedene Timer wiederum eigenständig. Diese kann manuell auf 0.5 reduziert werden. In einem Linux-basierenden Betriebssystem hängt die Resolution von der Software Clock ab. Dabei wird die Zeit in sog. \textbf{Jiffies} gemessen. Die Linux-Kernel version 2.6.0 führt eine jiffiegröße von 0.001 sekunden ein\footpartcite{mantime}. Dementsprechen sind die gemessenen Zeitstempel mit diesen Hintergrundinformationen zu bewerten.

Ein Span besitzt einen Spankontext und eine Referenzliste. Der Inhalt des Spankontext wird in \cref{subsection:SpanContext} beschrieben. Die Referenzliste beinhaltet alle Spans, die eine \textbf{Happens-Before Relation} zu dem Span aufweisen. Das bedeutet, das alle Elternspans, beispielsweise aus \emph{Child-Of} und \emph{Follows-From} resultierenden Beziehungen, in dieser Liste referenziert werden.

Schlussendlich kennt ein Span seinen Tracer. Dies ist erforderlich, da beim fertigstellen des Spans die \emph{Finish} Funktion aufgerufen wird. Diese sorgt dafür, das der Endzeitstempel gesetzt wird. Ausserdem wird die \emph{Report} Funktion der Reporterinstanz, bekannt durch den Tracer, aufgerufen.

\subsection{SpanBuilder}
\label{subsection:SpanBuilder}

Der SpanBuilder ist die Bindegliedentität zwischen dem Tracer und der Spangenerierung. Der Anwender der Instrumentalisierungsbibliothek kann durch die \emph{BuildSpan} Funktion des Tracer die Spangenerierung einleiten. Dabei wird eine SpanBuilder Instanz erstellt, die die übergebenen Parameter zur Spangenerierung nutzt. Der SpanBuilder verlangt bei der Initialisierung nach einem Operationsnamen und der Referenz des aufrufenden Tracers. Diese werden als Felder in der SpanBuilder Instanz gespeichert. Durch die Funktionen \emph{AddReference} und \emph{AsChildOf} kann das Feld \emph{references} bearbeitet werden. Dabei ist AsChildOf eine spezialisierte Form der AddReference Funktion. Die AddReference Funktion erwartet einen Referenztypen und einen SpanContext als Paramenter. Diese werden dann für den zu bauenden Span genutzt. 

Eine weitere Grundfunktion ist die \emph{Start} Funktion. Diese beinhaltet die nötige Anwendungslogik, für die Traceidentifikationsnummergenerierung. Auch die Spanidentifikationsnummer wird generiert. Anschließend wird der Span gebaut und mit den gesammelten Daten initialisiert.

Auch von der Start Funktion gibt es Spezialisierungen. Die überladene \emph{StartActive} Funktion kann Parameterlos oder mit dem Boolean \emph{finishSpanOnDispose} aufgerufen werden. Der Boolean bestimmt, wie mit einem Span umgegangen wird, nachdem die entsprechende Finish Funktion aufgerufen worden ist. StartActive setzt den gebauten Span auf den Zustand \textbf{Aktiv}. Das bedeutet, dass der Scopemanager diesen verfolgt und gegebenfalls den vorherig aktiven Span zwischenspeichert.

Eine beispielhafte Nutzung des Spanbuilders sieht folgendermaßen aus:

\begin{minipage}[]{\textwidth}
	\begin{lstlisting}[frame=trBL]
	var operationname = "example_function";
	var spanBuilder = tracer.BuildSpan(operationname);
	var span = spanBuilder.Start();
	\end{lstlisting}
	\captionof{lstlisting}{Beispielhafte Anwendung des SpanBuilder}
	\label{listing:SpanBuiler}
\end{minipage} 

Ein Operationsname wird definiert und als Paramenter für die BuildSpan-Funktion genutzt. Anschließend wird durch die Start-Funktion der SpanBuilder-Instanz ein Span auf den Zustand Aktiv gesetzt.


\subsection{SpanContext}
\label{subsection:SpanContext}

Die SpanContext-Klasse beinhaltet eine TraceId, eine SpanId und einen Rerenztypen als Felder. Diese werden bei der Initialisierung gesetzt. Die SpanContext-Klasse ist ein reine Datenkapselung. Semantisch sind die in der SpanContext enthaltenen Daten jene, die in einem BinaryCarrier, codiert als Bytearray, über die Leitung gesendet werden.

Ein Auszug eines Spankontexts aus dem Entwicklungssystem zeigt den Aufbau. Der Auszug stammt aus dem Registryservice. In dem Service wird der Nachricht auf die Konsole ausgebenen, sobald eine Propagiererung ausgelöst worden ist.

\begin{minipage}[]{\textwidth}
	\begin{lstlisting}[frame=trBL]
	traktor-registry_1 | Client-Message: WIoJiNwldhLM;ZtDdH4lQ1a1b;child_of
	\end{lstlisting}
	\captionof{lstlisting}{Ein über die Registry propagierter Spankontext}
	\label{listing:SpanContext-Registry}
\end{minipage} 

Der Spankontext beinhaltet drei Daten. Eine TraceId, eine SpanId und einem Relationstyp.

\subsection{Scope}
\label{subsection:Scope}

Ein Scope ist lokaler Kontext, welcher von dem ScopeManager verwaltet wird. In einem Scope werden Spans verwaltet. Dadurch wird das in \cref{subsection:Tracingcontext innerhalb eines Systems} beschriebene Problem der Kontextverfolgbarkeit innerhalb eines Systems gelöst. Ein Scope wird in einem Prozess geführt. Innerhalb des Scopes wird bei der aktiven Spans bei seiner Ersetzung gespeichert. Dadurch ist bei einer Beendigung des aktiven Spans der vorherige Span wiederherstellbar. Traktor nutzt die CSharp Implementierung des OpenTracing-Projekts.

\subsection{ScopeManager}
\label{subsection:ScopeManager}
Ein ScopeManager verwaltet Scopes. Bei einem Kontextwechsel, zum Beispiel bei einer Multi-Thread Implementierung, werden die Scopes durch den Manager verfolgt. Ein Scope ist nur innerhalb seines Prozesses relevant und beinhaltet einen aktiven Span. Über den Spanmanager lässt sich der Span, welcher durch einen aktiven Scope umfasst wird, ermitteln. Der Tracer ist dadurch jederzeit in der Lage, den aktiven Span und seinen Kontext zu nutzen. Wie auch der Scope, wird die ScopeManager-Implementierung des OpenTracing-Projekts genutzt. 

\subsection{Reporter}
\label{subsection:Reporter}
Die Reporter-Klasse kapselt die UDP-Verbindung des Tracers zu dem Agentenservice. Dabei wird die Netzwerkddresse, sowie der Port gespeichert. Bei der Initialisierung des Reporters, wird die Verbindung mit den gegebenen Parametern hergestellt.

Die Reporter-Klase umfasst drei Funktionen. Die Funktion \emph{Connect} stellt mit gegebenen Parametern die Verbindung zu dem UDP-Service her. Der UDP-Klient wird als Feld gespeichert und kann zur weiteren Datenübertragung genutzt werden. Die Funktion \emph{Report} sendet kodierte Form des Spans zu dem Agentenservice. Die Kodierung wird durch die Funktion \emph{BuildMessage} implementiert. Traktor nutzt eine UTF-8 Kodierung für den Nachrichtenaustausch.

\subsection{UML-Klassendiagramm der Bibliothek}
\label{subsection:UML-Klassendiagramm der Bibliothek}

Die Instrumentalisierungsbibliothek ist in der Programmiersprache CSharp implementiert. Die Programmiersprache eignet sich für die Entwicklung von Systemkomponenten, die in einer verteilten System zum Einsatz kommen, wie aus der Spezifikation der Programmiersprache entnommen werden kann: 

\begin{quote}
	\cbstart
	C\# ist gedacht als simple, moderne,  universal einsetzbare, objektorientierte Programmiersprache. [...] Die Sprache ist für die Entwicklung von Softwarekomponenten gedacht, die in einer verteilten Umgebung bereitgestellt werden.\footpartcite[S. xx]{10.5555/861332}
	\cbend
\end{quote}

Das Klassendiagramm, gezeigt in \cref{fig:TraktorKlassendiagramm}, stellt einen Überblick über die vorherig beschriebenen relevanten Entitäten der Bibliothek dar.

\newpage
\begin{landscape}
	\begin{figure}
		\centering
		\includegraphics[scale=0.4]{img/Implementierung/TraktorKlassendiagramm.png}
		\caption[Klassendiagramm der Traktor Instrumentalisierungsbibliothek]{Klassendiagramm der Traktor Instrumentalisierungsbibliothek}
		\label{fig:TraktorKlassendiagramm}
	\end{figure}
\end{landscape}

\section{Traktor Agent}
\label{section:Traktor Agent}

Der Traktoragent ist ein Service, welcher ein UDP-Endpunkt bereit stellt. Der Service gibt alle erhaltenen Nachrichten auf seiner Konsole aus. Zwei Umgebungsvariablen werden genutzt, um den UDP-Endpunkt zu initialisieren. Die \textbf{UDP\_IP} ist der \emph{Localhost}, da der Service auf einer eigenständigen Komponeten bereitgestellt wird. Dies ist durch den Spezifikikationspunkt \textbf{TE}.2  gegeben. Der belegete Port wird durch die Umgebungsvariable \textbf{UDP\_PORT} konfiguriert. 

Ein reporteter Span kann folgendermaßen aussehen:

\begin{minipage}[]{\textwidth}
	\begin{lstlisting}[frame=trBL]
	traktor-agent_1| recieved message: 
		 b'Process Context;
		 04/14/2020 10:10:06.1791 PM;
		 WIoJiNwldhLM;ZtDdH4lQ1a1b;
		 child_of;
		 04/14/2020 10:10:06.4158 PM'
	traktor-agent_1| from:  ('172.22.0.5', 13338)
	\end{lstlisting}
	\captionof{lstlisting}{Ein reportetet Span. Der gezeigte Span ist in der Webserver Entwicklungsumgebung generiert worden}
	\label{listing:Reporteter-Span}
\end{minipage} 

Der Service \emph{traktor-agent\_1} erhält von der Netzwerkaddresse 172.22.0.5:13338 die dargestellte Nachricht eines Spans mit dem Operationsnamen \emph{Process Context}. Der Startzeitstempel und der Endzeitstempel sind Teil des reporteten Spans. Auch der Spankontext ist dargestellt.

\section{Traktor Registry}
\label{section:Traktor Registry}

Die Traktor Registry ist ein Websocket Server. Verbindungsanfragen, initiiert von einer Tracerinstanz, werden als eigenständiger Thread in eine Klientenliste gespeichert der Registry gespeichert. Die Anwendungslogik der \emph{ClientHandler}-Threads implementiert die Websocket Handshakes. Das Websocketprotokoll wird in \cref{subsection:Websocketprotokoll} beschrieben. Die Kontextpropagierung wird durch den Websocket Server umgesetzt. Die Implementierung wird in \cref{subsection:Klientenverwaltung} vorgestellt.

\subsection{Websocketprotokoll}
\label{subsection:Websocketprotokoll}

Das Websocketprotokoll ermöglicht eine vollduplex Kommunikation zwischen der Registry und dem entsprechenden Websocketclient eines Tracers. Das Protokoll nutzt dafür eine einzige TCP-Verbindung. Das Websocketprotokoll wird genutzt, um einen geringen \textbf{Overhead} der Kommunikation zu gewährleisten. Dementsprechend wird das Designziel der \textbf{Verarbeitungskosten} berücksichtigt. Das Websocketprotokoll ist durch \gls{rfcLabel} mit der Nummer 6455 spezifiziert.

Das Websocketprotokoll verlangt einen öffnenden Handshake zur Etablierung der Verbindung. Dieser basiert auf einem HTTP Handshake. Die eröffnende HTTP-Anfrage beinhaltet eine \emph{Upgrade}-Anforderung der in diesem Moment genutzten HTTP-Verbindung. Der folgende Logausschnitt zeigt eine solche Eröffnungsnachricht:

\begin{minipage}[]{\textwidth}
	\begin{lstlisting}[frame=trBL]
	GET / HTTP/1.1
	Host: traktor-registry:8090
	Connection: Upgrade
	Upgrade: websocket
	Sec-WebSocket-Version: 13
	Sec-WebSocket-Key: s4VKefPmikGz1rJ24buoaQ==
	\end{lstlisting}
	\captionof{lstlisting}{Eröffnende Nachricht eines Websocket Handshake}
	\label{listing:Eröffnender Websocket Handshake}
\end{minipage} 

Die HTTP-Nachricht beinhaltet sogenannte \textbf{Header}, die Daten beinhalten, die für den Protokollwechsel von HTTP zu Websocket nötig sind. Der Host-Header identifiziert den Ursprung des Handshakeinitiators. Dieser ist in diesem Falle die IP-Addresse die hinter dem Alias \emph{traktor-registry} steht. Der Port 8090 der Anwendung, von welchem die Anfrage stammt, wird zusätzlich mitgesendet. Der Connection-Header gibt die bervorzugte Verbindungsart an. Diese beschreibt den gewünschten Vorgang eines Upgrades der Verbindung. In Verbindung mit dem Inhalt des Connections-Headers wird ein Upgrade-Header gesendet. Dieser ist ein Vorschlag an den Server ein anderes Protokoll zu nutzen. Entsprechend beinhaltet der Upgrade-Header den Vorschlag das Websocket Protokoll zu verwenden. Die letzen beiden Header Sec-WebSocket-Version und Sec-WebSocket-Key sind websocketspezifische Header.

Die Serverantwort sieht ähnlich aus. Diese besteht aus der Request-Line und dem Sec-WebSocket-Accept Header, der eine Bestätigung der zu nutzenden Websocketverbindung darstellt.

\begin{minipage}[]{\textwidth}
	\begin{lstlisting}[frame=trBL]
	HTTP/1.1 101 Switching Protocols
	Upgrade: websocket
	Connection: Upgrade
	Sec-WebSocket-Accept: HSmrc0sMlYUkAGmm5OPpG2HaGWk=
	\end{lstlisting}
	\captionof{lstlisting}{Serverantwort eines Websocket Handshake}
	\label{listing:Serverantwort eines Websocket Handshake}
\end{minipage} 

\subsection{Klientenverwaltung}
\label{subsection:Klientenverwaltung}



\section{Unity Rendering System}
\label{section:Unity Rendering System}
\section{ Webserver Entwicklungsumgebung }
\label{section:Webserver Entwicklungsumgebung}
% ----------------------------------------------------------------------------

% ----------------------------------------------------------------------------
% Copyright (c) 2016 by Burkhardt Renz. All rights reserved.
% Die Vorlage für eine Abschlussarbeit in der Informatik am Fachbereich
% MNI der THM ist lizenziert unter einer Creative Commons
% Namensnennung-Nicht kommerziell 4.0 International Lizenz.
%
% Id:$
% ----------------------------------------------------------------------------

\chapter{Evaluierung}
\label{chapter:Evaluierung}
In diesem Kapitel werden die Prozesse beschrieben, die durchgeführt wurden, um die Implementierung und die erhaltenen Ergebnisse zu evaluieren. In \cref{section:Anforderungserfüllung} wird die Traktorimplementierung auf ihre Einhaltung der Anforderungen  untersucht. In \cref{section:Umsetzung der Designziele} werden die Designziele herangezogen und auf ihre Gegebenheit in der Tracingbibliothek überprüft. Das Open-Source Projekt \textbf{Jaeger} wird als Vergleichswerkzeug in den Folgenden Abschnitten herangezogen. Jaeger ist eine auf der OpenTracing API basierenden \emph{state-of-the-art} Distributed Tracing Implementierung. Sie setzt die aktuellsten Erkenntnisse der Distributed Tracing Gemeinschaft um. Dabei wird in \cref{section:Bereitstellung der Testumgebung} die Bereitstellung der Testumgebung, in Hinsicht auf beide Werkzeuge, diskutiert. Die Bereitstellungsunterschiede beider Werkzeuge werden aufgezeigt. In \cref{section:Ergebnissvergleich} werden die Ergebnisse der Spangenerierung verglichen. Es wird auf die Ausdruckskraft des Traktor-Datenmodells im Vergleich zu Jaeger eingegangen. Zuletzt werden die präsentierten Visualisierungansätze diskutiert. Es wird ein Vergleich zu den Visualisierungsmöglichkeiten der Jaeger UI durchgeführt.

\section{Anforderungserfüllung}
\label{section:Anforderungserfüllung}

Es ist eine Analyse durchzuführen, bei der die erhobenen Daten interpretiert werden. Die daraus gewonnenen Informationen sollen die End-zu-End Latenz einer Anfrage durch ein verteiltes System und die Generierungszeit eines Frames, welches durch die Unity Anwendung generiert wurde, darstellen.
\begin{itemize}
	\item Funktionalitäten
		\begin{itemize}
			\item Eventgenerierung
			\item Eventrelation
			\item Synchronisation von Eventgeneratoren
			\item Eventübermittlung
			\item Ordnung von Events
		\end{itemize}
	\item End-zu-End Latenz
	\item Generierungszeit eines Frames
	\item Nachrichtenmodifikation
\end{itemize}

\section{Umsetzung der Designziele}
\label{section:Umsetzung der Designziele}
\section{Bereitstellung der Testumgebung}
\label{section:Bereitstellung der Testumgebung}
\begin{itemize}
	\item Durch Docker-Compose wird eine verteiltes System auf einem Lokalen Rechner aufgebaut.
	\item gibt es architektonische unterschiede? Services etc. 
	\item Konfigurationsunterschiede
	\item Ist traktor einfacher zu deployen? Wahrschienlich nicht da jaeger ein all in one image hat. Könnte aber auch umgesetzt werden
\end{itemize}
\section{Ergebnissvergleich}
\label{section:Ergebnissvergleich}


\begin{itemize}
	\item Bezug zur Problemstellung schaffen
	\item inwiefern hält sich jaeger an die Happens before relationship?
	\item wie setzt jaeger diese um?
	\item setzt es sie um?
	\item setzt traktor sie um? Wenn ja wie?
	\item Wie regelt Jaeger die Zeitresolution?
	\item  
\end{itemize}
\section{Visualisierungvergleich von Traktor und Jaeger}
\label{section:Visualisierungvergleich von Traktor und Jaeger}
\begin{itemize}
	\item jaeger bietet verschiedene visualisierungsmöglichkeiten
	\begin{itemize}
		\item z.B. DAG
		\item Tracediagramm
		\item Service-Orientierter Ansatz?
		\item Tracecompare
	\end{itemize}
	\item Welche Vorteile bringen meine Visualisierungsansatze?
	
\end{itemize}
% ----------------------------------------------------------------------------

% ----------------------------------------------------------------------------
% Copyright (c) 2016 by Burkhardt Renz. All rights reserved.
% Die Vorlage für eine Abschlussarbeit in der Informatik am Fachbereich
% MNI der THM ist lizenziert unter einer Creative Commons
% Namensnennung-Nicht kommerziell 4.0 International Lizenz.
%
% Id:$
% ----------------------------------------------------------------------------

\chapter{Fazit}
Die Hypothese, die in der Zielsetzung definiert worden ist, besagt, dass die Implementierungen der Inversenberechnung in C bessere Laufzeiten aufweisen, als in R.

Dies kann in Anbetracht der Messergebnis nur Teilweise bestätigt werden. 
\begin{itemize}
	\item Bei kleinen Matrizen trifft dies definitiv zu, wie in der Ergebnissdiskussion aufgezeigt wurde.
	\item Bei immer größer werdenden Matrizen steigt die Performance von R im Verhältnis zu den C-Varianten.
\end{itemize}

Diese Arbeit hat jedoch auch deutlich gemacht, wie stark die Performance von einer Implementierung abhängt. Es wurden gleiche Algorithmen mit unterschiedlichen Implementierungen verglichen und es hat sich herausgestellt, dass deutliche Unterschiede festzustellen sind. 

Die beiden naiven Ansätze leiden an gefährlichen Fehlern, wie z.B. Speicherlecks, die zum Absturz des Programms führen können. Das bedeutet, dass man bei der Auswahl seiner Werkzeuge, unbedingt die Rahmenbedingungen miteinbeziehen sollte. Damit ist gemeint, dass falls man die Möglichkeit hat, bewährte Packete einzusetzten, dann ist es empfohlen dies zu tun, anstatt das Rad neu zu erfinden.

Die C-Varianten können vorallem bei dem Speicherbedarf glänzen. Dafür sind jedoch hoher Entwicklungsaufwand und extreme Vorsicht bei der Implementierung aufzubringen.

Die R-Variante bietet eine enorme Auswahl an mathematischen Funktionen, die meistens sehr perfomant sind. Der Entwicklungsaufwand von Code ist ungemein geringer.

In Anbetracht der heutigen Ressourcen kann man in vielen Anwendungsfällen beruhigt auf die einfacheren Scriptsprachen zurückgreifen. Es lässt sich aber in speziellen Fällen nicht umgehen, die volle Kontrolle, bei z.B. sicherheitskritischen Bereichen, zu übernehmen. In diesem Fall bietet sich C an.
% ----------------------------------------------------------------------------

\backmatter 

\appendix
% ----------------------------------------------------------------------------
% Copyright (c) 2016 by Burkhardt Renz. All rights reserved.
% Die Vorlage für eine Abschlussarbeit in der Informatik am Fachbereich
% MNI der THM ist lizenziert unter einer Creative Commons
% Namensnennung-Nicht kommerziell 4.0 International Lizenz.
%
% Id:$
% ----------------------------------------------------------------------------


%\begin{thebibliography}{99}
\printbibliography
%\bibitem{Ritchie78}
%	Dennis M. Ritchie, Brian W. Kernighan
%	\emph{The C Programming Language}1st edition,
%	Reading, MA: Prentice Halls,  ISBN 0-13-110163-3,
%	1978.

%\end{thebibliography}

% ----------------------------------------------------------------------------



\end{document}
% ----------------------------------------------------------------------------
