\newacronym{fpsLabel}{FPS}{Frames per Second}
\newacronym{MMULabel}{MMU}{Memory Mangagment Unit}
\newacronym{cpuLabel}{CPU}{Central Processor Unit}
\newacronym{gpuLabel}{GPU}{Graphical Processor Unit}
\newacronym{ramLabel}{RAM}{Random-Access Memory}
\newacronym{hddLabel}{HDD}{Hard Disk Drive}
\newacronym{tcpLabel}{TCP}{Transmission Control Protocol}
\newacronym{ipLabel}{IP}{Internet Protocol}
\newacronym{udpLabel}{UDP}{User Datagramm Protocol}
\newacronym{httpLabel}{HTTP}{Hypertext Transfer Protocol}
\newacronym{rfcLabel}{RFC}{Request for Comments}
\newacronym{iotLabel}{IoT}{Internet of Things}
\newacronym{apiLabel}{API}{Application Programming Interface}
\newacronym{cncfLabel}{CNCF}{Cloud Native Computing Foundation}
\newacronym{guidLabel}{GUID}{Globally Unique Identifier}
\newacronym{lfLabel}{LF}{Linux Foundation}
\newglossaryentry{cpuGlossar}{
	name={Zentrale Verarbeitungseinheit},
	description={Die zentrale Verarbeitungseinheit, im englischen auch \emph{Central Processor Unit} (CPU) genannt, ist die Komponente eines Systems, welches für die Abarbeitung des Maschinenprogramms zuständig ist. Im Allgemeinen Sprachgebraucht umfasst der Begriff der CPU mehrere Teilkomponenten, wie zum Beispiel dem Steuerwerk, der Register, des Speichermanagers (\gls{MMULabel})}}
\newglossaryentry{ciGlossar}{
	name={Continuous Integration},
	description={Automatisierter und fortlaufend getätigter Integrationsprozess von Änderungen einer Anwendungen.}}
\newglossaryentry{renderingGlossar}{
	name={Rendering},
	description={
	Rendering ist das erzeugen von zwei-dimensionalen Bildern.
	Rohdaten, wie z.B. virtuelle Kameras, drei-dimensionale Objekte, Lichtquellen, etc. werden dazu genutzt, um diese Bilder zu generieren}
}
\newglossaryentry{tddGlossar}{
	name={Test-driven development},
	description={
		Test-driven development ist ein Software Entwicklungskonzept. Kurze Iterationsphase stehen durch Testimplementierung im Vordergrund. Eine Testbasis senkt die technischen Schulden und erleichtert eine automatisierte Bereitstellung. 
	}
}

